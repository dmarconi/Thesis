\section{Physic Object Recostruction}

\begin{table}[htb]
  \caption{definition of Vertex.}
   \label{table:vertexobjdefinition}
  \ttfamily\scriptsize\selectfont
  \begin{center}
    \begin{tabular}{|l|ll|}
      \hline
      \multicolumn{3}{|l|}{collection \texttt{label: recoVertex}}\\
      \multicolumn{3}{|l|}{type: \texttt{offlinePrimaryVertices}}\\
      \hline
      vertex.size() & $>$ & 0 \\
      \hline
    \end{tabular}
  \end{center}
\end{table}

\subsection{Jet Reconstruction}

Particle Flow jets (PFJets) are used in this analysis.PFJets use information from all subdetectors to produce a mutually exclusive collection of particles (namely muons, electrons, photons, charged hadrons and neutral hadrons) that are used as input for the jet clustering algorithms. The \antikt clustering algorithm with a reconstruction cone of R = 0.5 is used, defined in $\eta-\phi$ ($R = \sqrt{{\Delta \eta}^2 + {\Delta \phi}^{2}}$) \cite{1126-6708-2008-04-063}. The PFJets used in this analysis are corrected using L1 FastJet, L2 Relative, and L3 Absolute corrections. The L1 FastJet corrections use the event-by-event UE/PU (UE: Underlying Event) densities to remove the additional contributions to the measured jet energies due to underlying event and pile-up particles. The L2 and L3 corrections use jet balancing and $\gamma+$ Jet events to improve and provide a better energy response as a function of \pt and $\eta$. This analysis uses the "loose" working point Jet-Id selection criteria. Table~\ref{table:bjetobjdefinition} shows the selection criteria used for the recommended "loose" working point. The efficiency is $> 98$ for the entire $\eta$ and \pt range. The "loose'' working point has been validated in other studies.

\begin{table}[htb]
  \caption{definition of jets.}
   \label{table:jetobjdefinition}
  \begin{center}
    \ttfamily\scriptsize\selectfont
    \begin{tabular}{|l|ll|}
      \hline
      \multicolumn{3}{|l|}{ collection label: selectedPatJets}\\
      \multicolumn{3}{|l|}{ type: \texttt{pat::Jet}}\\
      \hline
      jet.pt() & $>=$ & 30. \\
      fabs(jet.eta()) & $<=$ & 5.0 \\
      jet.neutralHadronEnergyFraction() & $<$ &  0.99 \\
      jet.neutralEmEnergyFraction() & $<$ & 0.99 \\
      jet.numberOfDaughters() & $>$& 1 \\
      if(fabs(jet.eta()) $<$ 2.4) && \\
      ~~~jet.chargedHadronEnergyFraction() & $>$ & 0 \\
      ~~~jet.chargedEmEnergyFraction() & $<$ & 0.99 \\
      ~~~jet.chargedMultiplicity() & $>$ & 0 \\
      DeltaR(jet,tau) & $>=$ & 0.3 \\
      \hline
    \end{tabular}
  \end{center}
\end{table}

\subsubsection{b-Jet Tagging}
B-tagged jets are used to reduce \ttbar background in the signal region and to obtain $t\bar{t}$ enriched control samples used to estimate the signal rate. 
This analysis uses the "loose" working point of the combined secondary vertex algorithm.
The details of the algorithm can be found in \cite{CMS_PAS_BTV_11-001}.The EPS13 prescription is used for the b-tagging and mis-tagging scale factors and efficiencies. They are applied using the method called ``Event reweighting using scale factors only" \url{https://twiki.cern.ch/twiki/bin/viewauth/CMS/BTagSFMethods\#1c\_Event\_reweighting\_using\_scale}. Table~\ref{table:bjetobjdefinition} shows the selection criteria.

\begin{table}[htb]
  \caption{definition of $b$-jets.}
  \label{table:bjetobjdefinition}
  \begin{center}
  \ttfamily\scriptsize\selectfont
  \begin{tabular}{|l|ll|}
    \hline
    \multicolumn{3}{|l|}{ collection label: selectedPatJets}\\
    \multicolumn{3}{|l|}{ type: \texttt{pat::Jet}}\\
    \hline
    jet.pt() & $>=$ &  30. \\
    fabs(jet.eta()) & $<=$ & 2.4 \\
    DeltaR(jet,tau) & $>=$ & 0.3 \\
    jet.bDiscriminator(?) & $>$ & 0.244 \\
    \hline
  \end{tabular}
  \end{center}
\end{table}
\subsection{Tau Reconstruction and Identification}

The challenge in identifying hadronically decaying taus is discriminating against generic quark and gluon QCD jets which are produced with a cross--section several orders of magnitude larger. CMS has developed several algorithms to reconstruct and identify hadronically decaying taus based on Particle Flow (PF) objects. For this analysis, the tau POG recommended Hadron Plus Strips algorithm (HPS) is used. HPS makes use of PF jets as inputs to an algorithm that uses strips of clustered electromagnetic particles to reconstruct neutral pions. The electromagnetic strips ("neutral pions'') are combined with the charged hadrons within the PFJets to attempt to reconstruct the main tau decay modes outlined in Table~\ref{table:taumodes}.

\begin{table}[ht]
  \caption{Reconstructed Tau Decay Modes}
   \label{table:taumodes}
  \centering{
  \begin{tabular}{| c |}
  \hline\hline
        HPS Tau Decay Modes \\ [0.5ex] \hline
        Single Hadron + Zero Strip \\
        Single Hadron + One Strip \\
        Single Hadron + Two Strips \\
        Three Hadrons \\
  \hline
  \hline
  \end{tabular}
  }
\end{table}

The single hadron plus zero strips decay mode attempts to reconstruct $\tau \to \nu\pi^{\pm}$ decays or $\tau \to 
\nu\pi^{\pm}\pi^{0}$ decays where the neutral pion has very low energy. The single hadron plus one or two 
electromagnetic strips attempts to reconstruct tau decays that produce neutral pions where the resulting neutral pion 
decays produce collinear photons. Similarly, the single hadron plus two strips mode attempts to reconstruct taus that 
decay via e.g. $\tau \to \nu\pi^{\pm}\pi^{0}$ where the neutral pion decays to well separated photons resulting in two 
electromagnetic strips. The three hadrons decay mode attempts to reconstruct tau decays that occur via 
$\rho(770)$ resonance. In all cases, electromagnetic strips are required to have $E_{T} > 1$ GeV/c. Additionally,
the particle flow charged hadrons are required to be compatible with a common 
vertex and have a net charge of $|q|=1$.

In order to enforce the isolation requirement on the reconstructed tau, a region of size R = 0.5 around the tau decay 
mode direction is defined. Any PF candidates not used for the reconstruction of electromagnetic strips and charged 
hadrons not involved in the reconstruction of the tau decay modes are used to calculate isolation. The "Tight" mva isolation (with lifetime) working points are used to define the signal regions.

In order to discriminate against muons, HPS taus are required to pass the lepton rejection 
discriminator which requires the lead track of the tau not be associated with a global muon signature. In order to 
discriminate against electrons, HPS taus are required to pass a MVA discriminator which uses the amount of HCAL energy 
associated to the tau with respect to the measured momentum of the track (H/p). Additionally, the MVA discriminator 
considers the amount of electromagnetic energy in a narrow strip around the leading track with respect to the total 
electromagnetic energy of the tau. Finally, HPS taus must not reside in the ECAL cracks. 
In all channels, the identification and isolation used follows the Tau POG recommended criteria.
The exact discriminator names and working points for each channel are listed and described in their respective sections. Table~\ref{table:bjetobjdefinition} shows the selection criteria for $\tau$ leptons.


\begin{table}[htb]
  \caption{definition of $\tau$ leptons.}
  \label{table:tauobjdefinition}
  \begin{center}
  \ttfamily\scriptsize\selectfont
  \begin{tabular}{|l|ll|}
    \hline
    \multicolumn{3}{|l|}{ collection label: \texttt{patTaus}}\\
    \multicolumn{3}{|l|}{ type: \texttt{pat::Tau}}\\
    \hline
    fabs(tau.eta()) & $<=$ & 2.1 \\
    tau.pt() & $>=$ & 45.0 \\
    tau.leadPFChargedHadrCand()-$>$pt() & $>=$ & 5.0 \\
    tau.tauID(``byTightIsolationMVA3newDMwLT'') & $>$ & 0.5 ||\\
    ~tau.tauID(``byMediumIsolationMVA3newDMwLT'') & $>$ & 0.5 ||\\
    ~tau.tauID(``byLooseIsolationMVA3newDMwLT'') & $>$ & 0.5 \\
    (decayModeFindingNewDMs & $>$ & 0.5 $\&\&$ \\ 
    ~signalPFChargedHadrCands().size() & $==$ & 1) \\
    tau.tauID(``againstElectronMediumMVA5'') & $>$ & 0.5 \\
    tau.tauID(``againstMuonLoose3'') & $>$ & 0.5 \\
    \hline
  \end{tabular}
  \end{center}
\end{table}

\begin{table}[htb]
  \caption{definition of \met}
  \label{table:metobjdefinition}
  \ttfamily\scriptsize\selectfont
  \begin{center}
   \begin{tabular}{|l|ll|}
      \hline
      \multicolumn{3}{|l|}{ collection \texttt{label: patMET}}\\
      \multicolumn{3}{|l|}{ type: \texttt{patPfMetT0pcT1Txy}}\\
      \hline
    \end{tabular}
  \end{center}
\end{table}

\section{Physic Object Recostruction at 13 TeV}