

\section{Signal and background samples}

\begin{itemize}
	\item List of samples used for the studies
	\item brief details on how the signal samples has been generated and how data is stored now (miniaods)
	
\end{itemize}

\section{Event selection}

\begin{itemize}
	\item Different trigger choice $\rightarrow$  Lower tau pt selection;
	\item tau reconstruction commissioned down to 20 \gev (chosen variable for cuts optimization);
	\item MET cut removed (chosen variable for cuts optimization)
	\item Dijet delta eta cut removed (obsolete because strictly correlated to the $m_{jj}$ cut) 
	\item $m_{jj}$ cut removed (chosen variable for cuts optimization)
\end{itemize}

\section{Sensitivity and cross section limit studies}

\begin{itemize}
	\item Optimization of cuts to exclude signal at the lowest cross section;
	\item the reference study formula is $\dfrac{S}{\sqrt{B + (0.5 \cdot B)^{2}}} > 2$
	\item signal events can also be express as: $S = \epsilon ( Pt_{\tau} , m_{jj} ,  \met )\sigma L$
	\item The sensitivity study is done as function of 3 different variables: $\tau_{pt}$, MET and $m_{jj}$
	\item Background prediction in SR: two-fold ABCD method involving 2 different correction factors
\end{itemize}

