\section {Introduction}

Many of the searches for \charginopm and \neutralinotwo in ATLAS \cite{Aad:2012hba, ATLAS:2012ab} and CMS \cite{Chatrchyan:2012mea} exploit events with three leptons and \met. Those searches dealt with some difficulties. Firstly, with the increasing luminosity at the LHC both experiments needed to raise the \pt thresholds on the triggered objects which degraded signal efficiency. Secondly Drell-Yan production cross-section of sleptons steeply declines with an increasing invariant mass \cite{Baer:1997nh}.

As shown in early studies, however, an analysis of the \charginopm / \neutralinotwo system from a different angle  in Vector Boson Fusion (VBF) events is also possible \cite{Bjorken:1992er}. VBF production is characterized by the presence of two jets with a large di-jet invariant mass in the forward region in opposite hemispheres. Additionally the produced \charginopm and \neutralinotwo decay into multiple $\tau$ leptons and a \neutralinoone which travels through the detector undetected anc can only be reconstructed as \met. A sample diagrams for \charginopm / \neutralinotwo pair production from VBF processes are shown in Figure \ref{fig:VBF_diagrams}. 

\begin{figure}[tbh!]
	\centering
	\begin{tabular}{cc}
		\includegraphics[width=0.48\textwidth]{diagrams/pics/signal_C1N2.pdf}
		\includegraphics[width=0.48\textwidth]{diagrams/pics/signal_C1C1.pdf} 		
	\end{tabular}
	\caption{Diagrams of (left) \charginopm \neutralinotwo and (right) \charginopm \charginomp pair production through vector-boson fusion including their decays to $\tau$ leptons and the LSP.}
	\label{fig:VBF_diagrams}
\end{figure}

\section{VBF with two leptons and two jets}

This search for SUSY events with two leptons and two jets has some advantages with respect to the previous ones \cite{Dutta:2012xe}. Firstly, there is the possibility to probe signal for SUSY by triggering over the VBF properties of the event. This means that low kinematic constraints on the decay products of the \charginopm \charginopm pairs have to be made.

Secondly VBF production allows the investigation of final states with $\tau$ leptons. This can be of advantage for SUSY scenarios with high \tanbeta, in which case the \stau is typically lighter than  \smuon and \selectron  \cite{Hinchliffe:1999zc}. A light \stau with small mass splitting is favored in coannihilation processes \cite{Griest:1990kh} that set the relic density to correct values, in the case of Bino-like dark matter. A light \stau is also motivated in the context of the MSSM by the enhancement of the $H \longrightarrow \gamma\gamma$ channel \cite{Carena:2011aa}. These facts stress the importance of searches in $\tau$ final states with low \pt and large backgrounds, for which production by VBF processes is more suited since the VBF signature allows for the reduction of the backgrounds to manageable levels.

Finally, Drell-Yan production cross-section declines faster than the VBF production cross-section with increasing mass allowing further control over background distributions \cite{Datta:2002vy}.

The main feature of VBF processes is the production of a jet pair aimed at the forward-backward region of the detector with high \pt and large \deltaeta. By adding to the event selection the requirements on the di-jet \deltaeta as well as the di-jet invariant mass \ensuremath{m_{j_{1}j_{2}}} the background contribution coming from V+jets and \ttbar events, shown in Figures \ref{fig:background_W3jets} and \ref{fig:background_ttbar}, is kept under control. In order to generate supersymmetric particles, the incoming partons need to have an high momentum, so that the leading jet from the VBF-produced di-jet pair is expected to have high \pt. The addition of a \pt cut on leading jets further reduces background contributions. Figure \ref{fig:VBF_mjj_ptj1} shows a study on \ensuremath{m_{j_{1}j_{2}}} and leading jet \pt distributions for \charginopm \charginopm pair production by VBF processes, V+jets background, and VV background produced by VBF processes for \ensuremath{m_{\charginopm} = m_{\neutralinotwo} = 300, 200} and \ensuremath{100\gev}, \ensuremath{m_{\charginopm} - m_{\tilde{\tau}} = 5\gev} and \ensuremath{m_{\neutralinoone} = 0\gev}.

The remaining background contributions come from all the centrally produced particles. By considering an R-parity conserving model the decay of \charginopm and \neutralinotwo are:

\begin{equation}
\charginopm \longrightarrow \stau^{\pm} \nu \longrightarrow \tau^{\pm} \neutralinoone \nu ;
\end{equation}

\begin{equation}
\neutralinotwo \longrightarrow \stau^{\pm} \tau^{\mp} \longrightarrow \tau^{\pm} \tau^{\mp} \neutralinoone.
\end{equation}

Processes with same signature are all the VV (where V may be both W or Z) pairs produced via VBF where the bosons decays leptonically. Furthermore, in case of \hadtau decays Quantum Chromo Dynamics (QCD) multijet becomes the largest background contribution. A cut on \met is effective in reducing those backgrounds as shown on Figure \ref{fig:VBF_met_pttau}. Requiring multiple $\tau$ leptons in the event further reduces Standard Model background contributions. The \pt of the \ensuremath{\tau} coming from the \charginopm and \neutralinotwo decays is strongly correlated to the mass difference between the \charginopm and the \neutralinoone LSP. In Figure \ref{fig:VBF_met_pttau}, the normalized distribution of the \pt of \ensuremath{\tau} is displayed for \ensuremath{m_{\charginopm} = m_{\neutralinotwo} = 300\gev}, \ensuremath{m_{\charginopm} - m_{\tilde{\tau}} = 5\gev} and \ensuremath{m_{\neutralinoone} = 0\gev}. For smaller \ensuremath{\Delta M}, the distribution peaks at lower \pt and the signal acceptance is less efficient.

\section {Search Strategy}
\label{section::search_strategy}

For this type of search several benchmark points are defined with the following constraints. Firstly the \charginopm and \neutralinotwo are mainly Wino-like, while the \neutralinoone is mainly Bino-like. Furthermore, the \charginopm mass is similar to the \neutralinotwo mass ($m_{\charginopm} \sim m_{\neutralinotwo}$) and at values of 100, 200 and 300\gev. Additionally the mass gap between the \stau and \charginopm is either 5 \gev or $(m_{\stau} - m_{\charginopm})/2$. Finally The LSP mass is either $\neutralinoone = 0$, or 50\gev.

The following processes are taken into account:

\begin{equation}
pp \longrightarrow \charginopm \charginomp jj, \quad pp \longrightarrow \charginopm \neutralinotwo jj, \quad pp \longrightarrow \neutralinotwo \neutralinotwo jj
\end{equation}

The cross-section prediction for each of those processes are shown in Figure \ref{fig:VBF_xsec} as function of the \charginomp - \neutralinotwo mass.

\begin{figure}[tbh!]
	\centering
	\begin{tabular}{cc}
		\includegraphics[width=0.75\textwidth]{analysis/pics/VBFXsection.png}
	\end{tabular}
	\caption{VBF production cross-section at \CM = 8 \tev as a function of mass for various channels after imposing \ensuremath{\deltaeta > 4.2} using Madgraph 4 (NLO) \cite{Dutta:2012xe}.}
	\label{fig:VBF_xsec}
\end{figure}

The search strategy can be divided in two distinct parts: the first one considers the kinematic of the jets produced via VBF in order to reduce the contribution coming  from V + jets events (where V is either the W or Z boson); the second one focuses on the decay products of the supersymmetric particles falling into the inner region of the detector (centrally produced) in order to reduce the overall background contributions.

\begin{figure}[tbh!]
	\centering
	\begin{tabular}{cc}
		\includegraphics[width=0.48\textwidth]{analysis/pics/h_dijetinvariantmass_prospects13tev.pdf}
		\includegraphics[width=0.48\textwidth]{analysis/pics/h_jet1pt_prospects13tev.pdf} 		
	\end{tabular}
	\caption{\ensuremath{m_{j_{1}j_{2}}} (left) and \pt of the leading jet (right) distributions normalized to arbitrary units for \charginopm \charginopm pair production by VBF processes, V+jets background, and VV background produced by VBF processes and QCD processes. The chosen signal benchmark point features \ensuremath{m_{\charginopm} = m_{\neutralinotwo} = 300, 200} and \ensuremath{100\gev}, \ensuremath{m_{\charginopm} - m_{\tilde{\tau}} = 5\gev} and \ensuremath{m_{\neutralinoone} = 0\gev}.}
	\label{fig:VBF_mjj_ptj1}
\end{figure}

\begin{figure}[tbh!]
	\centering
	\begin{tabular}{cc}
		\includegraphics[width=0.48\textwidth]{analysis/pics/h_met_prospects13tev.pdf}
		\includegraphics[width=0.48\textwidth]{analysis/pics/h_tau1pt_prospects13tev.pdf} 		
	\end{tabular}
	\caption{(left) \met and (right) leading \ensuremath{\tau} \pt distributions normalized to arbitrary units in \ensuremath{\geq 2j + 2\tau} final state for \charginopm \charginopm pair production by VBF processes, V+jets background, and VV background produced by VBF processes and QCD processes. The chosen signal benchmark point features \ensuremath{m_{\charginopm} = m_{\neutralinotwo} =300, 200} and \ensuremath{100\gev}, \ensuremath{m_{\charginopm} - m_{\tilde{\tau}} = 5\gev} and \ensuremath{m_{\neutralinoone} = 0\gev}.}
	\label{fig:VBF_met_pttau}
\end{figure}


\section{VBF with two same sign hadronic $\tau$ and two jets}

Besides the two oppositely directed forward jets that define the VBF configuration, the search requires the presence of at least two leptons coming from the different decay modes of the $\tau$ and large \met. The chosen search channels are $e\mu jj$, $\mu\mu jj$, $\mu\hadtau jj$, and $\hadtau\hadtau jj$. The final states are further differentiated into like-sign (LS) and opposite-sign (OS) di-lepton pairs for a total of eight different search channels \cite{Khachatryan:2015kxa}. This thesis focuses its attention on the di-\hadtau LS channel search and its background estimation strategy. 


\subsection{Background Contributions}
\label{sec::bg_contributions}

The most important aspect of any search for new physics is the methodology used to estimate the background contribution in the signal region. The first step consists in the determination of all irreducible background processes and the definition of a technique to estimate each of the contributions. Four types of background contributions have been considered in this analysis. 

The first irreducible background contribution is coming from the Standard Model VBF processes resulting in two \hadtau and two jets as shown in Figure \ref{fig:background_SMVBF}. This background contribution is well modeled being purely electroweak and its cross section is very small. Therefore this background is considered minor and its contribution is directly estimated using Monte Carlo simulation.

\begin{figure}[tbh!]
	\centering
	\begin{tabular}{cc}
		\includegraphics[width=0.48\textwidth]{diagrams/pics/background_SMVBFminus.pdf}
		\includegraphics[width=0.48\textwidth]{diagrams/pics/background_SMVBFplus.pdf} 		
	\end{tabular}
	\caption{Feynman diagrams of irreducible backgrounds from Standard Model Vector Boson Fusion processes. Two hadronically-decaying $\tau$ leptons and two jets are observed. }
	\label{fig:background_SMVBF}
\end{figure}

Secondly, all the Standard Model VBF processes resulting in three leptons (Figure \ref{fig:background_SMVBFZ0Wmiss}), with the opposite charged lepton failing to pass the tauID.

\begin{figure}[tbh!]
	\centering
	\begin{tabular}{cc}
		\includegraphics[width=0.48\textwidth]{diagrams/pics/background_SMVBFZ0Wmissminus.pdf}
		\includegraphics[width=0.48\textwidth]{diagrams/pics/background_SMVBFZ0Wmissplus.pdf} 		
	\end{tabular}
	\caption{Feynman diagrams of irreducible backgrounds from Standard Model Vector Boson  processes. Three $\tau$ are produced and the opposite charged lepton fails to pass the tauID selection criteria. }
	\label{fig:background_SMVBFZ0Wmiss}
\end{figure}

Those processes are again purely electroweak, therefore the rate and relative theoretical uncertainty on the rate are small compared to QCD. Additionally, the probability for events coming from this background contribution to pass the event selection is very low. In conclusion, this topology of events is considered to be a minor contribution to the total background.  

Thirdly, there are the Standard Model backgrounds where one of the \hadtau has a mis-reconstructed charge as shown in Figures \ref{fig:background_SMVBFZ0}, \ref{fig:background_SMVBFH} and \ref{fig:background_ttbar}. All of these processes have a very low cross section compared to the one coming from the main background contribution, therefore those processes are all considered a minor background and their event contribution is taken directly from simulation.

\begin{figure}[tbh!]
	\centering
	\begin{tabular}{cc}
		\includegraphics[width=0.48\textwidth]{diagrams/pics/background_SMVBFZ0Z0.pdf}
		\includegraphics[width=0.48\textwidth]{diagrams/pics/background_SMVBFZ0W.pdf} 		
	\end{tabular}
	\caption{Feynman diagrams of irreducible backgrounds from Standard Model Vector Boson Fusion $Z^{0}$ production. }
	\label{fig:background_SMVBFZ0}
\end{figure}

\begin{figure}[tbh!]
	\centering
	\begin{tabular}{cc}
		\includegraphics[width=0.48\textwidth]{diagrams/pics/background_HZ0.pdf}
		\includegraphics[width=0.48\textwidth]{diagrams/pics/background_HW.pdf} 		
	\end{tabular}
	\caption{Feynman diagrams of irreducible backgrounds from Standard Model Vector Boson Fusion Higgs production. }
	\label{fig:background_SMVBFH}
\end{figure}

\begin{figure}[tbh!]
	\centering
	\begin{tabular}{cc}
		\includegraphics[width=0.48\textwidth]{diagrams/pics/background_ttbar.pdf}
		\includegraphics[width=0.48\textwidth]{diagrams/pics/background_singlet.pdf}
	\end{tabular}
	\caption{Feynman diagrams of \ttbar (left) and single top (right) production which can be a background process if one of the jets is reconstructed as a \hadtau.}
	\label{fig:background_ttbar}
\end{figure}

The last background contribution comes from all the QCD events resulting in four jets with two of those jets reconstructed as a fake \hadtau. Even though the probability of a jet to fake a \hadtau is low, the production cross sections of QCD events in a hadron collider are large. Therefore, even small \hadtaufake rates from QCD processes, matter. Also, in QCD events, additional jet activity from initial or final jet radiation processes is expected, which gives these type of events a high probability to pass the VBF selection criteria. Those motivations lead to the determination of QCD as main background source for this analysis. Examples of some QCD processes are shown in Figure \ref{fig:background_QCDinitrad}.

\begin{figure}[tbh!]
	\centering
	\begin{tabular}{cc}
		\includegraphics[width=0.48\textwidth]{diagrams/pics/background_QCDinitrad.pdf}
		\includegraphics[width=0.48\textwidth]{diagrams/pics/background_QCDfinrad.pdf}
	\end{tabular}
	\caption{Feynman diagram of a four jet QCD prodution with initial state radiation (left) and final state radiation (right) where two jets are misreconstructed as \hadtau. }
	\label{fig:background_QCDinitrad}
\end{figure}

Furthermore, there are QCD processes resulting in one \hadtau and three jets with one of the jets mis-reconstructed as the second \hadtau. Examples of those processes are shown on Figure \ref{fig:background_W3jets}

\begin{figure}[tbh!]
	\centering
	\begin{tabular}{cc}
		\includegraphics[width=0.75\textwidth]{diagrams/pics/background_W3jets.pdf}
	\end{tabular}
	\caption{Feynman diagrams of W boson plus 3 jets production where one jet is mis-reconstructed as \hadtau. }
	\label{fig:background_W3jets}
\end{figure}

\clearpage
