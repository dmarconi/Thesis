\section{Event Selection}
\label{sec:eventselection}

As mentioned in Section \ref{section::search_strategy} the di-\hadtau channel's main background contribution comes from QCD multijet, with a rate several orders of magnitude larger than the rate in other contributions.  Hence this search channel, more than any of the others, relies on the efficient background rejection. Fortunately, the VBF and \met selections provide the required background suppression.

All the collision data events passing the requirements of the triggers shown on Table \ref{table:triggerdefinition} are considered as the interesting events for offline analysis. As previously introduced in Section \ref{section::search_strategy}, the event selection criteria is divided in two distinct parts: the central part, which takes into account the LSP and the decay products of the multiple \hadtau, and the VBF part, which cuts over the kinematic properties of the jets coming from a VBF process. The main differences with the other VBF SUSY searches with final states to light leptons, are the substantially tighter \hadtau requirements targeting at the suppression of QCD jet background and the looser missing transverse energy (\met) requirement, to recover some of the signal acceptance lost due to the larger discriminator based on isolation $\hadtau$ \pt thresholds needed to stay efficient with respect to the trigger.  

The selected events are required to have at least two \hadtau candidates as defined in Section \ref{subsec::objsel_tau}. Higher multiplicity \hadtau are constrained by the trigger. The like sign \hadtau candidates with the highest \pt and separated from each other by a minimum \deltar = 0.3 are then chosen to form a di-\hadtau candidate. 
Further, to reduce top pair contamination the event is required not to have any jet identified as a b--quark jet by the b--tagging algorithms using the {\textit combined secondary vertex loose} (CSVL) working point. Only jets with \pt $\ge 30 \gev$ and separated from the taus in the di--\hadtau pairs by $\Delta R \ge 0.3$ are searched for b--tags. The higher \pt cut of 30 $\gev$ on b-jets (a looser veto requirement than other analyses with light leptons) allows us to be more efficient with respect to the signal since the higher \pt threshold on taus reduces the contamination of $t\overline{t}$ to a large extent. Further, the event is required to have at least 30 \gev of \met. All the above described selections is what will be referred as {\textit central selections}.

Subsequently, the following event-wide requirements are imposed. The {\textit {VBF selections}} are imposed by requiring at least two jets as defined in Section \ref{subsec::objsel_jet}. Only jets separated from the leptons in the \hadtau\hadtau pair by $\deltar \ge 0.3$ are considered. All jet candidates passing the above requirements and having $\vert \Delta\eta \vert \ge 4.2$ and $\eta_{1}\cdot\eta_{2} < 0$ are combined to form di-jet candidates. The final and the most important of the requirement is an invariant mass of the di-jet candidate, namely \mjj, above the threshold of 250 \gev. In order to increase the event acceptance the analysis code algorithm takes into account every possible di-jet candidate combination and chooses the one that passes the VBF requirements and has the highest \mjj. 

For better visualization and understanding all the selection criteria are summarized the following way:

\begin{itemize}
	\item \textbf{Central selection}
	\begin{itemize}
		\item Trigger: \texttt{HLT\_DoubleMediumIsoPFTau35\_Trk*\_eta2p1\_Prong1\_v*}
		\item two one-prong hadronically decaying $\tau$ with $\pt\geq45~$\gev 
		\item $\met > $ 30
		\item at least two jets with $p_{T}^{jet}\geq30~$\gev, $|\eta_{jet}|\leq5$ and loose jetID
		\item $\Delta R(jet,\tau)\geq0.3$
		\item b-tag veto
	\end{itemize}
	\item \textbf{VBF selection}
	\begin{itemize}
		\item $|\Delta\eta(jet,jet)| > 4.2$
		\item $sign(\eta^{jet 1}\cdot\eta^{jet 2})==-1$
		\item $\mjj>250~$\gev
	\end{itemize}
\end{itemize}

\clearpage


\section{Background Contributions}
The most important aspect of any search for new physics is the methodology used to estimate the background contribution in the signal region. The first step consists in the determination of all the irreducible background sources and the definition of an estimation technique for each of the contributions. Four types of background contributions have been considered in this analysis. 

The first irreducible background contribution is coming from the Standard Model VBF processes resulting in two \hadtau and two jets as shown in Figure \ref{fig:background_SMVBF}. This background contribution is well modeled being purely electroweak and its cross section is very small. Therefore this background is considered minor and its contribution is directly taken from simulation.

\begin{figure}[tbh!]
	\centering
	\begin{tabular}{cc}
		\includegraphics[width=0.48\textwidth]{diagrams/pics/background_SMVBFminus.pdf}
		\includegraphics[width=0.48\textwidth]{diagrams/pics/background_SMVBFplus.pdf} 		
	\end{tabular}
	\caption{Feynman diagrams of irreducible backgrounds of Standard Model Vector Boson Fusion processes with two \hadtau and two jets final state. }
	\label{fig:background_SMVBF}
\end{figure}

Secondly, all the Standard Model VBF processes resulting in three leptons, with one of the leptons, the opposite charged one, failing to pass the tauID, are considered as shown on \ref{fig:background_SMVBFZ0Wmiss}.

\begin{figure}[tbh!]
	\centering
	\begin{tabular}{cc}
		\includegraphics[width=0.48\textwidth]{diagrams/pics/background_SMVBFZ0Wmissminus.pdf}
		\includegraphics[width=0.48\textwidth]{diagrams/pics/background_SMVBFZ0Wmissplus.pdf} 		
	\end{tabular}
	\caption{Feynman diagrams of irreducible backgrounds of Standard Model Vector Boson  processes ending with three $\tau$ where the opposite charged one fails to pass the tauID selection criteria. }
	\label{fig:background_SMVBFZ0Wmiss}
\end{figure}

Those processes are again purely electroweak, therefore their cross section is low and the simulation modeling is reliable. Additionally the probability for events coming from this background contribution to pass the event selection is very low. In conclusion, this topology of events is considered to be a minor contribution to the total background.  

Thirdly, there are the Standard Model backgrounds where one of the \hadtau has a mis-reconstructed charge as shown in Figures \ref{fig:background_SMVBFZ0}, \ref{fig:background_SMVBFH} and \ref{fig:background_ttbar}. 

\begin{figure}[tbh!]
	\centering
	\begin{tabular}{cc}
		\includegraphics[width=0.48\textwidth]{diagrams/pics/background_SMVBFZ0Z0.pdf}
		\includegraphics[width=0.48\textwidth]{diagrams/pics/background_SMVBFZ0W.pdf} 		
	\end{tabular}
	\caption{Feynman diagrams of irreducible backgrounds of Standard Model Vector Boson Fusion $Z^{0}$ production. }
	\label{fig:background_SMVBFZ0}
\end{figure}

\begin{figure}[tbh!]
	\centering
	\begin{tabular}{cc}
		\includegraphics[width=0.48\textwidth]{diagrams/pics/background_HZ0.pdf}
		\includegraphics[width=0.48\textwidth]{diagrams/pics/background_HW.pdf} 		
	\end{tabular}
	\caption{Feynman diagrams of irreducible backgrounds of Standard Model Vector Boson Fusion Higgs production. }
	\label{fig:background_SMVBFH}
\end{figure}

\begin{figure}[tbh!]
	\centering
	\begin{tabular}{cc}
		\includegraphics[width=0.75\textwidth]{diagrams/pics/background_ttbar.pdf}
	\end{tabular}
	\caption{\ttbar production where both jets fail the b-tag and one of the jets is reconstructed as a \hadtaufake.}
	\label{fig:background_ttbar}
\end{figure}

All of these processes have a very low cross section compared to the one coming from the main background contribution, therefore those processes are all considered a minor background and their event contribution is taken directly from simulation.

The last background contribution comes from all the QCD events resulting in 4 jets with 2 of those jests reconstructed as a fake \hadtau. Even though the probability of a jet to fake an \hadtau is low the production of QCD events in a hadron collider has huge cross sections. Therefore, even small \hadtaufake rates from QCD processes, matter. Also, in QCD events, additional jet activity from initial or final jet radiation processes is expected, which gives these type of events a high probability to pass the VBF selection criteria. Those motivations lead to the determination of QCD as main background source for this analysis. Examples of some QCD processes are shown in Figure \ref{fig:background_QCDinitrad} and \ref{fig:background_QCDfinrad}.

\begin{figure}[tbh!]
	\centering
	\begin{tabular}{cc}
		\includegraphics[width=0.75\textwidth]{diagrams/pics/background_QCDinitrad.pdf}
	\end{tabular}
	\caption{Feynman diagrams of a four jet QCD prodution with initial state radiation where two jets are misreconstructed as \hadtau. }
	\label{fig:background_QCDinitrad}
\end{figure}

\begin{figure}[tbh!]
	\centering
	\begin{tabular}{cc}
		\includegraphics[width=0.75\textwidth]{diagrams/pics/background_QCDfinrad.pdf}
	\end{tabular}
	\caption{Feynman diagrams of a 4 jet QCD prodution with final state radiation where two jets are misreconstructed as \hadtau. }
	\label{fig:background_QCDfinrad}
\end{figure}

Similarly to the processes described before there are QCD processes resulting in one \hadtau and three jets with one of the jets mis-reconstructed as the second \hadtau. Exaples of those processes are shown on Figure

\begin{figure}[tbh!]
	\centering
	\begin{tabular}{cc}
		\includegraphics[width=0.75\textwidth]{diagrams/pics/background_W3jets.pdf}
	\end{tabular}
	\caption{Feynman diagrams of W boson plus 3 jets production where one jet is mis-reconstructed as \hadtau. }
	\label{fig:background_W3jets}
\end{figure}

\begin{figure}[tbh!]
	\centering
	\begin{tabular}{cc}
		\includegraphics[width=0.75\textwidth]{diagrams/pics/background_singlet.pdf}
	\end{tabular}
	\caption{Feynman diagrams of single top production where one jet is mis-reconstructed as \hadtau. }
	\label{fig:background_singlet}
\end{figure}

\clearpage

\section{Signal and Background Samples}

The background yields are taken from simulation. Simulated samples of signal and background events are generated using Monte Carlo (MC) event generators. The signal event samples are generated with the \texttt{MadGraph v5.1.5} program \cite{Alwall:2011uj}, considering pair production of gauginos with two associated partons. The signal events are generated requiring a pseudorapidity gap $|\deltaeta| > 4.2$ between the two partons, with $\pt > 30$ \pt for each parton. Background event samples with a Higgs boson produced through VBF processes, and single top are generated with the \texttt{POWHEG v1.0r1380} program \cite{Frixione:2007vw}. The \texttt{MadGraph v5.1.3} generator is used to describe Z+jets, W+jets, tt, di-boson, and VBF Z boson production. The MC background and signal yields are normalized to the integrated luminosity of the data. The \ttbar background is normalized to the next-to-next-to-leading-logarithm level using the calculations of references \cite{Czakon:2013goa,Melnikov:2006kv}. The Z+jets and W+jets processes are normalized to next-to-next-to-leading-order using the results from the \texttt{FEWZ v2.1} \cite{Gavin:2010az} generator. The di-boson background processes are normalized to next-to-leading-order using the \texttt{MCFM v5.8} \cite{Campbell:2010ff} generator, while the VBF Z boson events are normalized to next-to-leading order using the \texttt{VBFNLO v2.6} \cite{Arnold:2008rz,Arnold:2011wj}program. The single-top and VBF Higgs boson background yields are taken from the powheg program, where the next-to-leading order effects are incorporated. Signal cross sections are calculated at leading order using the MadGraph generator. All MC samples incorporate the \texttt{CTEQ6L1} \cite{Pumplin:2002vw} or \texttt{CTEQ6M} \cite{Nadolsky:2008zw} parton distribution functions (PDF). The corresponding evaluation of uncertainties in the signal cross sections is discussed in Section \ref{sec:systematics}. The range of signal cross sections is $50^{–1}$ fb for \charginopm = \neutralinotwo masses of 100–-300\gev. The \texttt{POWHEG} and \texttt{MadGraph} generators are interfaced with the \texttt{PYTHIA v6.4.22} \cite{Sjostrand:2006za} program, which is used to describe the parton shower and hadronization processes. The decays of $\tau$ leptons are described using the \texttt{tauola (27.1215)} \cite{Davidson:2010rw} program. The background samples are processed with a detailed simulation of the CMS apparatus using the \texttt{Geant4} package \cite{Agostinelli:2002hh}, while the response for signal samples is modeled with the CMS fast simulation program \cite{Abdullin:2011zz}. For the signal acceptance and \mjj shapes based on the fast simulation, the differences with respect to the \texttt{Geant4}-based results are found to be small ($< 5\%$). Corrections are applied to account for the differences. For all MC samples, multiple proton-proton interactions are superimposed on the primary collision process, and events are reweighted such that the distribution of reconstructed collision vertices matches that in data. The distribution of the number of pileup interactions per event has a mean of 21 and a root-mean-square of 5.5. For all datasets, a common \texttt{Physics Analysis Toolkit} (\texttt{PAT}) \cite{Adam:2010zza} sequence has been used to generate a PAT format, then further reduced in size by the ntuple producer \cite{bib:thentuplemaker}.

\section {LS di-Tau Data-Driven QCD background prediction} \label{sec:bgestimation}

The QCD background contribution for the like-sign di-\hadtau channel is done using a ABCD data-driven approach. This method consist in  dividing the analyzed data in different exclusive regions, defined by two variables. The first variable is the \hadtau isolation discriminator used in the object reconstruction, namely:
 	
 	\begin{enumerate}
 		\item Tight or T isolated $\hadtau$ for \texttt{byTight\-IsolationMVA3newDMwLT};
 		\item Medium or M isolated $\hadtau$ for \texttt{byMedium\-IsolationMVA3newDMwLT} but failed \texttt{byTight\-IsolationMVA3newDMwLT};
 		\item Loose or L isolated $\hadtau$  for \texttt{byLoose\-IsolationMVA3newDMwLT} but failed \texttt{byTight\-IsolationMVA3newDMwLT} and \texttt{byMedium\-IsolationMVA3newDMwLT}.
 	\end{enumerate}
 	
Each event with a successfully reconstructed LS di-$\hadtau$ pair falls into a exclusive isolation region as shows on Figure \ref{fig:tauisoregions}:
 	
 	\begin{itemize}
 		\item SR or signal region consisting of two tight isolated $\hadtau$;
 		\item 1T or One Tight isolated $\hadtau$ region consisting of one tight isolated $\hadtau$ and an additional medium or loose isolated $\hadtau$;
 		\item AT or Anti Tight isolation region consisting of at least one medium isolated $\hadtau$ and an additional medium or loose isolated $\hadtau$;
 		\item AM or Anti Medium isolation region consisting of two loose isolated $\hadtau$.
 	\end{itemize}
 
 	\begin{figure}[tbh!]
 		\centering
 		\begin{tabular}{cc}
 			\includegraphics[width=0.75\textwidth]{PLOTS/diTauHadLSotherPlots/tauisoregions.png}
 		\end{tabular}
 		\caption{Definitions of the exclusive isolation region depending on the isolation of each of the $\hadtau$ where SR is Signal Region consisting of two tight isolated $\hadtau$, 1T is One Tight isolated Tau region consisting of one tight isolated $\hadtau$ and an additional medium or loose isolated $\hadtau$, AT is Anti Tight isolation region consisting of at least one medium isolated $\hadtau$ and an additional medium or loose isolated $\hadtau$,  AM is Anti Medium isolation region consisting of two loose isolated $\hadtau$}
 		\label{fig:tauisoregions}
 	\end{figure}
 
The second dimension of exclusivity is based on the VBF cuts described in \ref{sec:eventselection}. The regions are defined the following way:
	
	\begin{enumerate}
		\item VBF region: consisting of all the events that passed all VBF cuts previously mentioned;
		\item VBF inverted region: consisting of all the events that at least fails one of the VBF cuts previously mentioned;
	\end{enumerate} 

Using these definitions one signal region (SR) and seven control regions (CR) are defined as shown in Figure \ref{fig:crs}. The SR falls into the region defined by two tight isolated $\hadtau$ and all the VBF cuts applied, close to it the control region 2 (CR2) features the same \hadtau isolation requirements but inverted VBF selection requirement. In order to keep the QCD background contribution low in SR and CR2 an additional cut of  \met $ > $ 30\gev is required.

The estimation of the events in the signal region is done similarly to any other data driven ABCD method by counting the number of events in CR2 and multiplying it with a proper conversion factor described in the following equation:

\begin{equation}
N^{QCD}_{SR} = \left( N^{DATA}_{CR2} - N^{\overline{QCD} BG}_{CR2} \right) * \left[ \frac{\epsilon^{QCD}_{VBF}}{1 - \epsilon^{QCD}_{VBF}} \right]
\label{eq:qcdbgpred}
\end{equation}

where $N^{QCD}_{SR}$ is the number of QCD events predicted in the signal region, $N^{DATA}_{CR2}$ is the number of data events in CR2, $N^{\overline{QCD} BG}_{CR2}$ is the number of all non-QCD MC samples events in CR2 and $\epsilon^{QCD}_{VBF}$ is the efficiency of VBF cuts in a lower \hadtau isolation region. Those  control regions are defined as CR3 (One Tight isolation region), CR5  (Anti Tight isolation region) and CR7 (Anti Medium isolation region) followed by their corresponding VBF-inverted control regions CR4, CR6, CR8. An overview of the defined SR and CRs is shown on Figure \ref{fig:crs}. The $\epsilon^{QCD}_{VBF}$ for each of the different \hadtau isolation region is with the following equation.

\begin{figure}[tbh!]
	\centering
	\begin{tabular}{cc}
		\includegraphics[width=0.75\textwidth]{PLOTS/diTauHadLSotherPlots/controlregions.png}
	\end{tabular}
	\caption{Definition of Signal and Control Regions using different $\hadtau$ isolation criteria and VBF selection.}
	\label{fig:crs}
\end{figure}

\begin{equation}
\epsilon^{QCD}_{VBF} = \frac {N^{DATA}_{VBF CR} - N^{\overline{QCD} BG}_{VBFCR}}{\left( N^{DATA}_{VBFCR} - N^{\overline{QCD} BG}_{VBFCR} \right) + \left( N^{DATA}_{\overline{VBF}CR} - N^{\overline{QCD} BG}_{\overline{VBF}CR} \right) }
\label{eq:vbfeff}
\end{equation}

where $N^{DATA}_{VBF CR}$ is the number of the events in data for a given $ \tau $ isolation region and VBF region, $N^{\overline{QCD} BG}_{VBFCR}$ is the number of all non-QCD MC events for a given $ \tau $ isolation region and VBF region, $N^{DATA}_{\overline{VBF}CR}$ is the number of events in data in the same isolation Control Region but inverted VBF region and$N^{\overline{QCD} BG}_{\overline{VBF}CR}$ is the number of all non-QCD MC events for a given $ \tau $ isolation region but inverted VBF region.

This estimation method has been developed under the following assumptions:

\begin{itemize}
	\item[1] all control regions are QCD dominated;
	\item[2] $\epsilon^{QCD}_{VBF}$ is independent from any trigger efficiency concerning $\hadtau$ isolation such that each contribution to the numerator and denominator cancels out;
	\item[3] $\epsilon^{QCD}_{VBF}$ is independent from ant MET cut applied in order to reduce QCD background contributions. 
\end{itemize}

Using the events coming from the defined control regions as input for equations \ref{eq:qcdbgpred} and \ref{eq:vbfeff} gives the possibility to determine three different predictions for the QCD background contribution, one for each pair of tau isolation control regions below the two-tight isolation region. The estimation of the QCD contamination, in the signal region, as shown on Equation \ref{eq:qcdbgpred}, has three sources of systematics. The first source comes from the generated Monte Carlo samples used in the analysis. The two remaining ones comes from the assumptions about the stability of the $\epsilon^{QCD}_{VBF}$, made when defining the data-driven method, one in regard to a loosening of the tau-identification and the other in regard to a loosening of the \met cut, since the $\epsilon^{QCD}_{VBF}$ is calculated in CRs where no \met cut is applied.

Details and results on the validation process done for all the statements is given in Section \ref{QCD_bg_pred_validation}.

The signal and control regions are define under the assumption of central selection being orthogonal to VBF selection. Figure \ref{fig:LS_mjjshapestab_vs_tauiso_data} and \ref{fig:LS_mjjshapestab_vs_tauiso_mc} shows the results of the stability study of $M_{jj}$ shape distribution among different $\tau$ isolation sidebands for Data and MC. Further studies are shown in Section \ref{subsubsec:shapecomp}.

\begin{figure}[tbh!]
	\centering
	\begin{tabular}{cc}
		\includegraphics[width=0.75\textwidth]{PLOTS/diTauHadLSotherPlots/LS_mjjshapestab_vs_tauiso_data.pdf}
	\end{tabular}
	\caption{$M_{jj}$ shape comparisons among different $\tau$ isolation sidebands for Data (CR3, CR5, CR7)}
	\label{fig:LS_mjjshapestab_vs_tauiso_data}
\end{figure}

\begin{figure}[tbh!]
	\centering
	\begin{tabular}{cc}
		\includegraphics[width=0.75\textwidth]{PLOTS/diTauHadLSotherPlots/LS_mjjshapestab_vs_tauiso_mc.pdf}
	\end{tabular}
	\caption{$M_{jj}$ shape comparisons among different $\tau$ isolation sidebands for all MC contributions (CR3, CR5, CR7)}
	\label{fig:LS_mjjshapestab_vs_tauiso_mc}
\end{figure}



\begin{table}[ht]
	\centering{
		%   \tabcolsep=0.05cm
		\begin{tabular}{| l | c | c | c |}
			\hline\hline
			Sample     &Events (SR)   &Events (CR2)     &Events (CR3)       \\ [0.5ex] \hline
			Data &$ - $    &$ 109$    &$ 39$     \\
			Drell-Yan &$ 0.037\pm0.015$    &$ 1.3\pm1$    &$ 0.042\pm0.0077$     \\
			VV &$ 0.11\pm0.065$    &$ 0.7\pm0.09$    &$ 0.035\pm0.017$    \\
			W+Jets &$ 0.53\pm0.04$    &$ 6.6\pm0.17$    &$ 0.83\pm0.055$    \\
			Single t &$ 0.036\pm0.0066$    &$ 0.25\pm0.017$    &$ 0.057\pm0.008$   \\
			\ttbar &$ 0.11\pm0.012$    &$ 1.4\pm0.051$    &$ 0.19\pm0.013$   \\
			Higgs &$ 0.0005\pm7.2e-05$    &$ 0.012\pm0.0048$    &$ 0.0029\pm0.0023$   \\
			QCD &$ 8.6\pm0.61$    &$ 54\pm1.2$    &$ 47\pm1.9$   \\
			\hline
			Total nonQCD MC &$ 0.83\pm0.079$    &$ 10\pm1$    &$ 1.2\pm0.06$   \\
			\hline\hline
		\end{tabular}
	}
	\caption{Number on events in SR, CR2, CR3 for data and all MC samples used for the estimation of $N^{QCD}_{SR}$}
	\label{table:CReventcount1} % is used to refer this table in the text
\end{table}

\begin{table}[ht]
	\centering{
		%   \tabcolsep=0.05cm
		\begin{tabular}{| l | c | c | c |}
			\hline\hline
			Sample   &Events (CR4)  &Events (CR5)     &Events (CR6)    \\ [0.5ex] \hline
			Data    &$ 737$  &$ 22$    &$ 312$      \\
			Drell-Yan    &$ 0.65\pm0.045$  &$ 0.002\pm0.00076$    &$ 0.029\pm0.0037$    \\
			VV   &$ 0.6\pm0.1$   &$ 0.0031\pm0.001$    &$ 0.045\pm0.015$   \\
			W+Jets    &$ 10\pm0.2$  &$ 0.081\pm0.0096$    &$ 0.89\pm0.034$      \\
			Single t   &$ 0.47\pm0.02$   &$ 0.011\pm0.00076$    &$ 0.1\pm0.0028$      \\
			\ttbar  &$ 2.5\pm0.059$   &$ 0.04\pm0.0024$    &$ 0.52\pm0.0095$      \\
			Higgs  &$ 0.012\pm0.0044$   &$ 2.7e-05\pm1.1e-05$    &$ 0.00018\pm2.2e-05$    \\
			QCD  &$ 3.4e+02\pm4$  &$ 19\pm0.68$    &$ 1.4e+02\pm1.6$    \\
			\hline
			Total nonQCD MC  &$ 15\pm0.24$ &$ 0.14\pm0.01$    &$ 1.6\pm0.038$    \\
			\hline\hline
		\end{tabular}
	}
	\caption{Number on events in CR4, CR5, CR6 for data and all MC samples used for the estimation of $N^{QCD}_{SR}$}
	\label{table:CReventcount2} % is used to refer this table in the text
\end{table} 

\begin{table}[ht]
	\centering{
		%   \tabcolsep=0.05cm
		\begin{tabular}{| l | c | c |}
			\hline\hline
			Sample      &Events (CR7)     &Events (CR8)  \\ [0.5ex] \hline
			Data     &$ 17$    &$ 184 $  \\
			Drell-Yan    &$ 0.0012\pm0.00057$    &$ 0.013\pm0.0021 $  \\
			VV   &$ 0.0011\pm0.00014$    &$ 0.033\pm0.016 $  \\
			W+Jets    &$ 0.036\pm0.0051$    &$ 0.35\pm0.015 $  \\
			Single t   &$ 0.0061\pm0.00045$    &$ 0.058\pm0.00087 $  \\
			\ttbar  &$ 0.022\pm0.00079$    &$ 0.32\pm0.0054 $  \\
			Higgs  &$ 4.9e-06\pm2.3e-06$    &$ 5.7e-05\pm1e-05 $  \\
			QCD  &$ 15\pm0.81$    &$ 1.1e+02\pm1.7 $  \\
			\hline
			Total nonQCD MC  &$ 0.067\pm0.0053$    &$ 0.77\pm0.023 $  \\
			\hline\hline
		\end{tabular}
	}
	\caption{Number on events in CR7 and CR8 for data and all MC samples used for the estimation of $N^{QCD}_{SR}$}
	\label{table:CReventcount3} % is used to refer this table in the text
\end{table} 

\begin{table}[ht]
	\centering{
		\tabcolsep=0.05cm
		\begin{tabular}{| l | c | c | c |}
			\hline\hline
			Variable     &One Tight region     &Anti-Tight region     &Anti-Medium  \\ [0.5ex] \hline
			$\epsilon^{QCD}_{VBF}$    &$ 0.05\pm0.008 $  &$ 0.066\pm0.014 $  &$ 0.085\pm0.02 $ \\
			$N^{QCD}_{SR}$    &$ 5.2\pm1 $  &$ 6.9\pm1.7 $  &$ 9.1\pm2.5 $ \\
			\hline\hline
		\end{tabular}
	}
	\caption{ Values for $\epsilon^{QCD}_{VBF}$ and $N^{QCD}_{SR}$ for different $ \tau $ isolation regions.}
	\label{table:VBFeffBKGprediction} % is used to refer this table in the text
\end{table}

Systematics for the simulation are estimated by scaling non-QCD contributions by $\pm50~\%$. A systematic error on $\epsilon^{QCD}_{VBF}$ is assigned by using the maximal variation among all $\epsilon^{QCD}_{VBF}$ measurements in different $\tau$ isolation regions with respect to its weighted mean. Similar procedure is done for the assignation of the $\epsilon^{QCD}_{VBF}$ coming from \met cut stability. All the statistical uncertainties are propagated accordingly. Tables \ref{table:CReventcount1} , \ref{table:CReventcount2} and \ref{table:CReventcount3} show the event counting for all the control regions previously defined for data and all MC samples. All the numbers except the ones coming from the QCD sample are used as input for the QCD background estimation method. For each of the 3 different $\tau$ isolation regions out of the signal region (1T, AT, AM) an independent measurements of $\epsilon^{QCD}_{VBF}$  and prediction for $N^{QCD}_{SR}$ is made. Due to low statistics in the $\tau$ isolation sidebands the final results will include uncertainties coming from the $\epsilon^{QCD}_{VBF}$ stabilities studies on MC shown in section \ref{subsec:stability}.



\section{Data Driven method validation}
\label{QCD_bg_pred_validation}

While a data-driven approach with an ABCD method in data is utilizable in order to get a prediction for the number of background events expected in the signal region, there are two advantages of a simulation-based approach. On the one hand, one can ascertain that the control regions contain only QCD. On the other hand, one can provide a closure test for the data-driven prediction, especially regarding the universality of the VBF efficiency with respect to a loosening of the τh isolation or a relaxation of the \met requirement, as stated in Sec. ???.

In order to achieve the goal to ensure QCD purity in the control regions, a straightforward use of the simulation is not possible, due to the low probability of jets to fake \hadtau with any identification (ID), be it tight (T), medium (M), or loose (L). These low rates decrease the number of simulated events with two reconstructed \hadtau leptons passing all selection requirements below any reasonable amount for direct usage. What can be done, is to estimate the probability of one jet to be misidentified as a \hadtau lepton (O ≈ 1 ) of a given exclusive isolation (see Tab. ??? in Sec. ??? for definitions), 100 instead. Using all simulated events with four or more jets, one can determine the chance of each such event to be reconstructed as one event with two \hadtau leptons and at least two jets, translating two jet objects to become \hadtau objects (see Sec. ???). This way, a meaningful study of systematic uncertainties of the background estimation method on data becomes possible and the prerequisite assumptions of QCD purity and stability of the VBF efficiency with respect to \hadtaufake isolation can be tested. Finally, the results of the application of this method and thereby estimated systematic uncertainties will be shown (see Sec. ???). As well, possible future data-driven improvements for this method will be outlined (see Sec. ???).

\subsection*{Redefining jets as \hadtau}

\subsection*{Selection acceptance correction}

\subsection*{parametrization for fake probabilities}

\subsection*{Trigger acceptance correction}

\subsection*{Fixation of fake \hadtau mass}

\subsection{$\epsilon^{VBF}$-stability with regard to $\tau_{h}$-isolation and $E_{T}^{miss}$-cuts}\label{dihad:subsec:stability}
To test the stability and estimate the systematics, we determine on the diced MC all possible efficiencies of $\epsilon^{VBF}$ in all even and odd control region pairs for a range of $E_{T}^{miss}$-cuts in table \ref{dihad:tab:stability}.
\begin{table}[!h]
	\centering
	\begin{tabular}{|c||c|c|c|}
		\hline
		region     & $\epsilon^{VBF}(E_{T}^{miss}\geq0)$ [$\%$]& $\epsilon^{VBF}(E_{T}^{miss}\geq10)$ [$\%$]& $\epsilon^{VBF}(E_{T}^{miss}\geq20)$ [$\%$]\\ \hline \hline
		LS SR/CR2  & $12.78\pm0.53$ & $12.91\pm0.58$ & $13.06\pm0.67$ \\ \hline
		LS CR3/CR4 & $12.20\pm0.43$ & $12.48\pm0.47$ & $12.46\pm0.53$ \\ \hline
		LS CR5/CR6 & $11.61\pm0.41$ & $11.93\pm0.45$ & $11.91\pm0.50$ \\ \hline
		LS CR7/CR8 & $12.10\pm0.60$ & $12.44\pm0.66$ & $12.75\pm0.83$ \\ \hline \hline
		OS SR/CR2  & $11.46\pm0.72$ & $11.74\pm0.80$ & $12.23\pm1.04$ \\ \hline
		OS CR3/CR4 & $10.62\pm0.31$ & $10.66\pm0.33$ & $10.85\pm0.36$ \\ \hline
		OS CR5/CR6 & $10.98\pm0.48$ & $11.09\pm0.52$ & $11.59\pm0.67$ \\ \hline
		OS CR7/CR8 & $10.67\pm0.36$ & $10.61\pm0.37$ & $11.12\pm0.47$ \\ \hline \hline
		WAM        & $11.33\pm0.15$ & $11.43\pm0.16$ & $11.66\pm0.19$ \\ \hline
		\hline
		\hline
		region     & $\epsilon^{VBF}(E_{T}^{miss}\geq30)$ [$\%$]& $\epsilon^{VBF}(E_{T}^{miss}\geq40)$ [$\%$]& $\epsilon^{VBF}(E_{T}^{miss}\geq50)$ [$\%$]\\ \hline \hline
		LS SR/CR2  & $13.76\pm0.89$ & $14.99\pm1.49$ & $18.09\pm2.93$ \\ \hline
		LS CR3/CR4 & $13.18\pm0.68$ & $14.54\pm1.11$ & $16.58\pm2.08$ \\ \hline
		LS CR5/CR6 & $12.20\pm0.56$ & $13.06\pm0.84$ & $13.34\pm1.12$ \\ \hline
		LS CR7/CR8 & $14.05\pm1.25$ & $15.93\pm2.19$ & $19.26\pm4.14$ \\ \hline \hline
		OS SR/CR2  & $13.14\pm1.58$ & $16.04\pm2.98$ & $21.23\pm6.13$ \\ \hline
		OS CR3/CR4 & $11.08\pm0.41$ & $12.61\pm0.69$ & $14.45\pm1.23$ \\ \hline
		OS CR5/CR6 & $12.19\pm0.95$ & $14.38\pm1.70$ & $18.22\pm3.42$ \\ \hline
		OS CR7/CR8 & $11.37\pm0.54$ & $13.01\pm0.90$ & $14.76\pm1.62$ \\ \hline \hline
		WAM        & $11.98\pm0.23$ & $13.43\pm0.39$ & $14.85\pm0.65$ \\ \hline
	\end{tabular}
	\caption{Stability of epsilonVBF in regard to $E_{T}^{miss}$, sign and $\tau_{h}$-isolation. LS and OS regions are slightly correlated. Here, events can be the same, but the chosen jets to fake $\tau_{h}$ must differ. For efficiencies of one sign, efficiencies are expected to be highly correlated. This is not accounted for in the weighted arithmetic mean (WAM) values shown.}
	\label{dihad:tab:stability}
\end{table}
We do observe a quadratic dependence on $E_{T}^{miss}$, as seen in fig. \ref{dihad:fig:Stability}.

\begin{figure}[!h]
	\centering
	\includegraphics[width=0.5\textwidth]{PLOTS/diTauHadLSQCDPlots/stability/Stability.pdf}
	\caption{\label{dihad:fig:Stability}Deviation of the weighted arithmetic mean (WAM) of the VBF efficiency of all (LS and OS) control region ratios with respect to $E_{T}^{miss}$. We observe a quadratic dependence of small size at the cut-value of 30 GeV we use in this analysis.}
\end{figure}

We derive two systematic errors out of this:
\begin{enumerate}
	\item Stability of $\epsilon^{VBF}$ regarding $\tau_{h}$-isolation: Maximum relative difference to weighted arithmetic mean at $E_{T}^{miss}\geq30$. Amounts to $+17.26\%$ and $-7.58\%$.
	\item Stability of $\epsilon^{VBF}$ regarding $E_{T}^{miss}$-cut: Relative difference to $\epsilon^{VBF}$ within uncertainties of weighted arithmetic
	mean at no cut on $E_{T}^{miss}$ as seen in fig. \ref{dihad:fig:Stability}. Amounts to $+8.30\%$ on the upper edge and $+3.26\%$ on the lower edge.
\end{enumerate}

\clearpage



