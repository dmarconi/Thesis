Despite the Standard Model's success in describing subnuclear phenomena over the last decades, it is not a complete theory of particle physics. Many theories have been formulated in order to address the issues of this model. Among them, Supersymmetry has the potential to link gravity with the other fundamental forces of nature by proposing a relationship between two basic classes of elementary particles: bosons and fermions. Under Supersymmetry, each particle from one group has an associated particle in the other, called its superpartner, the spin of which differs by a half-integer. These superpartners would be new and undiscovered particles.

This thesis presents one of the first searches for supersymmetry in the vector-boson fusion topology. The search targets final states with at least two hadronically decaying tau leptons, large missing transverse momentum, and two jets with a large separation in pseudo-rapidity. The data sample corresponds to an integrated luminosity of 19.7\invfb of proton-proton collisions at a center-of-mass energy of 8\tev collected with the CMS detector at the CERN LHC. The observed di-jet invariant mass spectrum is found to be consistent with the expected standard model prediction. Upper limits are set on the cross sections for chargino and neutralino production with two associated jets, assuming the supersymmetric partner of the tau lepton to be the lightest slepton and the lightest slepton to be lighter than the charginos.

The second part of this thesis gives an outlook into the possible analysis strategies for the second phase of data taking. A sensitivity study is performed using 13\tev simulated data and an integrated luminosity of 85\invfb with the aim of establishing the most optimal and realistic event selection.
