Even though the Standard Model managed to successfully describe subnuclear phenomena over the last decades, it is not a complete theory of particle physics. Many are the theories that have been formulated in order to address the issues of this model. Among them Supersymmetry is a theory that links gravity with the other fundamental forces of nature by proposing a relationship between two basic classes of elementary particles: bosons and fermions. In Supersymmetry, each particle from one group would have an associated particle in the other, which is known as its superpartner, the spin of which differs by a half-integer. These superpartners would be new and undiscovered particles.

This thesis presents one of the first searches for supersymmetry in the vector-boson fusion topology. The search targets final states with at least two hadronically decaying tau leptons, large missing transverse momentum, and two jets with a large separation in rapidity. The data sample corresponds to an integrated luminosity of 19.7\invfb of proton-proton collisions at 8\tev of center-of-mass collected with the CMS detector at the CERN LHC. The observed dijet invariant mass spectrum is found to be consistent with the expected standard model prediction. Upper limits are set on the cross sections for chargino and neutralino production with two associated jets, assuming the supersymmetric partner of the tau lepton to be the lightest slepton and the lightest slepton to be lighter than the charginos.

The second part of this thesis gives an outlook into the analysis possible strategies for the second phase of data taking period. A sensitivity study is perfomed using 13\tev simulated data and an integrated luminosity of 85\invfb with the aim of searching for the most optimal and realistic event selection.
