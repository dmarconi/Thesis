The Standard Model of particle physics synthesizes our current understanding of nature \cite{Spiesberger:2000ks}. For decades, this model has managed to predict and lead to the discovery of several elementary particles and has provided a guiding framework for the experimental and theoretical scientific communities. 

Even after the many successful discoveries, most recently that of the Higgs Boson discovered in 2012 at the Large Hadron Collider \cite{Aad:2012tfa,Chatrchyan:2012xdj}, the Standard Model is not considered the final theory of particle physics. First of all, the theory is not complete. For example, it incorporates neither general relativity nor neutrino oscillations, and does not provide a dark matter candidate. Furthermore, the Standard Model suffers from a variety of problems. One of the most significant ones is the so-called hierarchy problem and relates to the Higgs boson mass, whitch has been measured at a value of around 125\gev and is considered very low regarding the huge radiative corrections at the Planck scale ($\approx 10^{19}\gev$). Many of these defects and tensions motivate both theoretical and experimental physicists to formulate theories beyond the Standard Model. One model in particular tries to solve the above mentioned problems through a new symmetry into the Lagrangian formulation of particle physics, a so-called supersymmetry (SUSY). This symmetry relates bosons and fermions by new fermionic generators and leads to the prediction of a supersymmetric partner particle for each of the particles contained in the Standard Model. If the new particles have mass near the electroweak scale, then would lead to very drastic phenomenological implications. A very large number of searches has been performed over the last decays in many high energy physics experiment. 

With the successful operation of the Large Hadron Collider, numerous results placing constraints on extensions to the Standard Model have been established by the ATLAS and CMS experiments. In particular, in SUSY models, limits in excess of 1\tev have been placed on the masses of the strongly produced gluinos and first and second generation squarks. In contrast, mass limits on the weakly produced charginos (\charginopm) and neutralinos (\neutralinotwo), with much smaller production cross sections, are much less severe. The limits for charginos and neutralinos are especially weak in so-called compressed mass spectrum, in which the mass of the lightest supersymmetric particle (LSP) is only slightly less than the masses of other SUSY states. The chargino-neutralino sector plays a crucial role in the connection between dark matter and SUSY: in SUSY models with R-parity conservation, the lightest neutralino \neutralinoone often takes the role of the LSP and is a dark matter candidate. Several searches on the chargino/neutralino system have been performed and showed a limited sensitivity in case the \charginopm and \neutralinotwo are nearly mass degenerate. Electroweak SUSY particles can be also produced in pairs along with two jets in in pure electroweak processes through vector-boson-fusion. This topology is characterized by the presence of two forward jets, in opposite hemispheres o fthe detector, leading to a large dijet invariant mass (\mjj). A search in the VBF topology offers a new and complementary means to directly probe the electroweak sector of SUSY, especially in compressed-mass-spectrum scenarios 

After an introduction to the theoretical and experimental background, this PhD thesis presents two different studies. The first part consist in a search for supersymmetric particles using 19.7 \invfb of proton-proton collision data, collected in the year 2012 at 8\tev of center-of-mass energy at the CMS detector. The second part shows a sensitivity and limit setting study performed with 85 \invfb of simulated data at 13\tev.
