Trotz des gro{\ss}en Erfolgs des Standardmodells bei der Beschreibung subatomare Ph\"anomene ist es keine vollst\"andige Theorie der Teilchenphysik. Viele neue Theorien wurden formuliert mit dem Ziel, Lösungsans\"atze f\"ur dessen M\"angel zu bieten.
Eine dieser neuen Theorien, Supersymmetrie, hat das Potential eine Verbindung zwischen der Gravitation und den anderen fundamentalen Kr\"aften der Natur herzustellen, indem ein Zusammenhang zwischen zwei Arten von Elementarteilchen, Fermionen und Bosonen, eingef\"uhrt wird. Dieser Zusammenhang besteht darin, dass jedes Teilchen aus der einen Gruppe mit einem sogenannten Superpartner aus der anderen Gruppe assoziiert ist, dessen Spin sich um eine Halbzahl unterscheiden.

Diese Arbeit beschreibt eine der ersten Suchen nach Supersymmetrie, die im Kanal mit Vektor-Boson-Fusion durchgef\"uhrt wurde. Die Suche zielt auf Endzust\"ande ab, bei denen mindestens zwei hadronisch zerfallenden Tau-Leptonen, hoher fehlender Transversalimpuls und zwei Jets mit gro{\ss}em r\"aumlichen Abstand bzgl. der Pseudo-Rapidit\"at vorhanden sind. Der verwendete Datensatz entspricht einer integrieren Luminosit\"at von 19.7\invfb, der bei Proton-Proton-Kollisionen mit einer Schwerpunktsenergien von 8\tev am CMS-Detektors am LHC gesammelt wurde. Die beobachtete Verteilung der Di-Jet-Masse ist konsistent mit der erwarteten Vorhersage durch das Standardmodell. Folglich wurden Ausschlussgrenzen auf den Wirkungsquerschnitt der Chargino- und Neutralino-Produktion in Assoziation mit Jets gesetzt, unter der Vorraussetzung, dass der supersymmetrische Partner des Tau-Leptons das leichteste Slepton ist und dieses wiederum leichter als die Charginos ist.

Der zweite Teil dieser Arbei gibt einen Ausblick in die m\"oglichen Analysestrategien f\"ur eine zweite Phase in der Datennahme bei einer Schwerpunktsenergie von 13\tev. Basierend auf einem simulierten Datensatz von 85\invfb wurde eine Sensitivit\"atsstudie durchgef\"uhrt, die das Ziel hat eine optimale und realistische Selektion von Ereignissen zu studieren.
