
This chapter covers the theoretical background for the research presented in this thesis. Starting with the description of the Standard Model, this chapter will analyze its limits and suggest one possible theoretical extension, supersymmetry. This additional theory can explain some of the phenomena in nature that the Standard Model cannot describe.

\section{The Standard Model of Particle Physics}

The Standard Model of particle physics is a theory that describes three of the forces existing in nature, the electromagnetic, the weak and the strong force and defines the fundamental constituents of matter \cite{Spiesberger:2000ks}. It was developed in the second half of the century as combined theoretical and experimental effort by the research communities from all around the world. Since the first formulation at the beginning of the 1970s the Standard Model successfully predicted all the particles that were discovered later in the century. The most recent particle discoveries such as the top quark (1995), the tauonic neutrino (2000) and finally the Higgs boson (2012) have given further credence to the Standard Model. 

According to this theoretical model the basic constituents of matter are leptons and quarks, which are divided into three families of identical structure. \autoref{table:fermions} shows all the fermions of the Standard Model and their charges, arranged in the three families. Details on the main aspects of this theory are gonna be given in the following sections.

\begin{figure}[tbh!]
	\begin{center}
		
		\begin{tabular}{ | c | c | c | c | c |}
			\hline
			& 1st Generation & 2 Generation & 3rd Generation & charge \\ \hline \hline
			& & & & \\
			leptons & $\left( \begin{array}{c} \nu_{e} \\ e \end{array} \right)_{\text{L}}$ & $\left( \begin{array}{c} \nu_{\mu} \\ \mu \end{array} \right)_{\text{L}}$ & $\left( \begin{array}{c} \nu_{\tau} \\ \tau \end{array} \right)_{\text{L}}$ & \begin{tabular}{@{}c@{}}weak \\ weak, electromagnetic\end{tabular} \\
			& & & & \\
			& $e_{\text{R}}$& $\mu_{\text{R}}$& $\tau_{\text{R}}$& electromagnetic\\ 
			& & & & \\
			\hline
			& & & & \\
			quarks & $\left( \begin{array}{c} u \\ d \end{array} \right)_{\text{L}}$ & $\left( \begin{array}{c} c \\ s \end{array} \right)_{\text{L}}$ & $\left( \begin{array}{c} t \\ b \end{array} \right)_{\text{L}}$ & weak, electromagnetic, strong \\
			& & & & \\
			& $u_{\text{R}}, d_{\text{R}}$& $c_{\text{R}}, s_{\text{R}}$& $t_{\text{R}}, b_{\text{R}}$& electromagnetic, strong\\
			& & & & \\ 
			\hline
			\hline
		\end{tabular}
		\caption{Fermions of the Standard Model and their charges, arranged in the three generations. Only the left-handed fermions interact weakly and are arranged in doublets. The right-handed fermions are singlets. The right-handed neutrinos are not present in this table, as they do not interact with one of the forces of the Standard Model.}
		\label{table:fermions}
	\end{center}
\end{figure}

\subsection{The Higgs Mechanism}
\label{higgs_mechanism}

The discovery of the Higgs boson led to the confirmation of a mechanism that was developed in 1964 to solve the problem of a symmetric Lagrangian with massless particles which did not agreed with experimental evidence. This mechanism works by introducing a new gauge boson and spontaneously breaking symmetry.

The starting point is moving from a global gauge transformation to a point-dependent gauge transformation applied to the minimal Lagrangian $\mathcal{L}$. This requires in addition to the starting complex scalar field $\phi$ also a vectorial field $A_{\mu}$ analog to the electromagnetic field. The resulting Lagrangian is:

\begin{equation}
\mathcal{L} = (D_{\mu}\phi)^{\dagger} D^{\mu}\phi - V (\phi) - \dfrac{1}{4}F_{\mu\nu}F^{\mu\nu}\
\label{eq::lagrangian_min}
\end{equation}

where $V(\phi)$ is the $\phi$ field potential:

\begin{equation}
V(\phi)=\mu^{2}\phi^{\dagger}\phi+\lambda(\phi^{\dagger}\phi)^{2} -\epsilon\phi^{\dagger} -\epsilon^{*}\phi
\end{equation}

$D^{\mu}$ is the covariant derivative:

\begin{equation}
D^{\mu} = \partial^{\mu} - ieA^{\mu},
\end{equation}

and $F^{\mu\nu}$ is the tensor of the vector field $A_{\mu}$:

\begin{equation}
F^{\mu\nu} =\partial^{\nu}A^{\mu} - \partial^{\mu}A^{\nu}.
\end{equation}

where $\epsilon \rightarrow 0$, $\mathcal{L}$ is invariant under gauge transformations:

\begin{equation}
\phi(x) \rightarrow e^{i\alpha(x)}\phi(x); \phi(x)^{\dagger} \rightarrow e ^{-i\alpha(x)}\phi(x)^{\dagger}
\end{equation}

\begin{equation}
A^{\mu} \rightarrow A^{\mu} + \dfrac{1}{e}\partial^{\mu}\alpha(x)
\label{eq::a_tranform}
\end{equation}

where $\alpha(x)$ is an arbitrary function of x and $e$ is a new coupling constant, identical to the electric charge in case $A_{\mu}$ is identified as the electromagnetic field. In order to obtain a stable theory $\lambda > 0$, however $\mu^{2}$ can have two cases depending on the sign choice:
 
1. $\mu^{2} > 0$. The energy minimum is at $\phi = 0$ and $A_{\mu} = 0$. The resulting theory consists of:

\begin{enumerate}
	\item a charged particle and its anti-partner, both with mass $\mu^{2} \neq 0$;
	\item a massless particle with spin 0, similar to the photon.
\end{enumerate}

For small values of $\lambda$ and $e$ the Lagrangian describes the interactions between scalar particles with the electromagnetic field (trough the coupling constant $e$) and their self-interactions (through the couping constant $\lambda$).

2. $\mu^{2} < 0$. The energy minimum is at $A_{\mu} = 0$, so in  case $\epsilon \rightarrow 0$:

\begin{equation}
V(\bar{\phi}) = \text{min }; \quad \bar{\phi}=\eta= 2\lambda +O(\epsilon)
\end{equation}
 
 in order to study the  variations around the minimum $\phi$ is defined as:

\begin{equation}
 \phi=\eta + \dfrac{\sigma_{1}(x) + i\sigma_{2}(x)}{\sqrt{2}} 
\end{equation}

and is used into the minimal Lagrangian \ref{eq::lagrangian_min}.

The particle masses spectrum is obtained from the fields $\sigma_{i}$ and $A_{\mu}$ quadratic terms. In addition, for $\alpha \rightarrow 0$, $\phi$ transforms as:

\begin{equation}
\phi \rightarrow \phi + i\alpha\phi = \eta + \dfrac{\sigma_{1}(x) + i\sigma_{2}(x)}{\sqrt{2}} + i\eta\alpha(x)
\end{equation}

or rather:

\begin{equation}
\sigma_{1}(x) \rightarrow \sigma^{\prime}_{1}(x) = σ_{1}(x); \quad \sigma_{2}(x) \rightarrow \sigma^{\prime}_{2}(x) = \sigma_{2}(x) + \sqrt{2}\eta\alpha(x)
\end{equation}

$A_{\mu}$ transforms following \autoref{eq::a_tranform}. The $\sigma_{2}$ field non-homogeneously transforms by the addition of an $\alpha$ term, arbitrary function of $x$. In case of the given $\sigma_{i}(x)$ and $A_{\mu}$ where:

\begin{equation}
\alpha(x) = -\dfrac{σ\sigma_{2}(x)}{\sqrt{2}\eta}
\end{equation}

resulting in:

\begin{equation}
\sigma^{\prime}_{2}(x) = 0
\label{eq::unitary_gauge}
\end{equation}

the $\sigma_{2}$ filed can be crossed out in case of a gauge symmetry. The gauge identified by \autoref{eq::unitary_gauge} is commonly known as unitary gauge.

\subsection{The bosonic masses}

The starting point is the electroweak Lagrangian $\mathcal{L}_{\text{eW}}$ based on the $\text{SU}(2)_{\text{L}} \oplus \text{U}(1)_{\text{Y}}$ symmetry:

\begin{equation}
\mathcal{L}_{\text{eW}} = \bar{l}i\gamma^{\mu} D_{\mu}l + \bar{e}_{\text{R}}i\gamma^{\mu}D_{\mu}e_{\text{R}} - \dfrac{1}{4}[W_{\mu\nu}W^{\mu\nu} +B_{\mu\nu}B^{\mu\nu}]
\label{eq::lagrangian_ew}
\end{equation}

with the leptonic fields defined as:

\begin{equation}
\binom{(\nu_{e})_{\text{L}}}{e_{\text{L}}}_{\text{Y}=−2} ; \quad (e_{\text{R}})_{\text{Y}=−2}
\label{eq::fields_scheme}
\end{equation}

and the covariant derivatives and tensors defined as:

\begin{equation}
D_{\mu}l = [\partial_{\mu} + igW_{\mu} \dfrac{\tau}{2}  + ig^{\prime}(-\dfrac{1}{2})B_{\mu}]l 
\end{equation}

\begin{equation}
D_{\mu}e_{\text{R}} = [\partial_{\mu} + ig^{\prime}(-1)B_{\mu}]e_{\text{R}}
\end{equation}

\begin{equation}
W_{\mu\nu} =\partial_{\nu}W_{\mu} - \partial_{\mu}W_{\nu} 
\end{equation}

\begin{equation}
B_{\mu\nu} =\partial_{\nu}B_{\mu} - \partial_{\mu}B_{\nu} 
\end{equation}

At this stage the theory describes massless fermions and vectorial fields. The introduction of a scalar field triggers a symmetry breaking while keeping the electromagnetism gauge still intact:

\begin{equation}
\text{SU}(2)_{\text{L}} \oplus \text{U}(1)_{\text{Y}} \rightarrow \text{U}(1)_{\text{em}}
\end{equation}

The optimal choice in order to generate the electron and quarks mass is to choose a $\text{SU}(2)_{\text{L}}$ doublet with $\text{Y} = +1$: 

\begin{equation}
\phi = \binom{(\phi{+})_{\text{L}}}{\phi^{0}}_{\text{Y}=+1}
\label{eq::su2_doublet}
\end{equation}

\begin{equation}
D_{\mu}\phi = [\partial_{\mu} + igW_{\mu} \dfrac{\tau}{2}  + ig^{\prime}(+\dfrac{1}{2})B_{\mu}]\phi 
\end{equation}


with the addition of the fully symmetric Higgs doublet into \autoref{eq::lagrangian_ew} the total Lagrangian becomes:

\begin{equation}
\mathcal{L}_{\text{tot}} = \mathcal{L}_{\text{eW}} + \mathcal{L}_{\phi W}
\end{equation}

where:

\begin{equation}
\mathcal{L}_{\phi W} = (D_{\mu}\phi)^{\dagger} (^{\mu}\phi) - V (\phi);
\label{eq::lagrangian_phiW}
\end{equation}

\begin{equation}
V(\phi) = \mu^{2}\phi^{\dagger}\phi + \lambda(\phi^{\dagger}\phi)^{2}
\end{equation}

Following the example shown in \autoref{higgs_mechanism}, $\phi$ can gain a vacuum expectation term, breaking the symmetry:

\begin{equation}
 \bar{\phi}= < 0|\phi|0 > = \binom{0}{\eta}
 \label{eq::vacuum_expectation}
\end{equation}

where:

\begin{equation}
\eta = \sqrt{\dfrac{-\mu^{2}}{2\lambda}}
\label{eq::eta_value}
\end{equation}

In \autoref{eq::su2_doublet} doublet the electric charge is represented as:

\begin{equation}
Q = 
\begin{pmatrix}
+1 & 0 \\
0 & 0 \\
\end{pmatrix}
\end{equation}

so that the field minimum is invariant under phase transformations associated to $\text{U}_{\text{em}}(1)$:

\begin{equation}
e^{i\alpha Q} \bar{\phi} = 
\begin{pmatrix}
e^{i\alpha} & 0 \\
0 & 1 \\ 
\end{pmatrix}
\binom{0}{\eta}
= \bar{\phi}
\end{equation}

The symmetry breaking given by $\bar{\phi} \neq 0$ gives the scheme \ref{eq::fields_scheme}.
In order to correctly identify all the particles qa definition of the unitary gauge conditions is mandatory. Knowing that every two-dimensional spinor can transform to a spinor with only a real lower component through a point-dependent gauge transformation. Therefore a given $\phi(x)$
can be defined as:

\begin{equation}
\phi(x) = \text{U}(x) \binom{0}{\rho(x)}
\end{equation}

with $\rho(x)$ real and $\text{U}(x)$ as matrix of $ \text{SU}(2)_{\text{L}} \oplus \text{U}(1)_{\text{Y}}$. 

In the previously defined unitary gauge the Lagrangian \ref{eq::lagrangian_phiW} becomes:

\begin{equation}
\begin{array}{r c l}
\mathcal{L}_{\phi W}&=&\dfrac{1}{2} \partial_{\mu} \sigma \partial^{\mu} \sigma - V \left[ \eta + \dfrac{\sigma(x)}{\sqrt{2}}\right] + g^{2} W_{\mu}^{i}(W^{j})^{\mu} \left[ \bar{\phi} \dfrac{\tau_{i}\tau_{j}}{4}\bar{\phi}\right] +\\
&+&(g^{\prime})^{2} \dfrac{1}{4}\eta^{2} B_{\mu} B^{\mu} + 2 gg^{\prime} W^{3}_{\mu} B^{\mu} \left[ \bar{\phi} \dfrac{\tau_{3}}{4} \bar{\phi} \right]
\end{array}
\end{equation}

By using  \autoref{eq::vacuum_expectation} and \autoref{eq::eta_value} along with the Pauli's matrixes properties:

\begin{equation}
\begin{array}{c}
 W_{\mu}^{i}(W^{j})^{\mu} \left[ \bar{\phi} \dfrac{\tau_{i}\tau_{j}}{4}\bar{\phi}\right] = \frac{1}{4}\eta^{2} W_{\mu}W^{\mu} \\

 W^{3}_{\mu} B^{\mu} \left[ \bar{\phi} \dfrac{\tau_{3}}{4} \bar{\phi} \right] = - \dfrac{1}{4} \eta^{2} W^{3}_{\mu}B^{\mu}
 \end{array}
\end{equation}

from the quadratic terms is possible to get the masses value:

\begin{equation}
\begin{array}{c}
W_{\mu}^{i}(W^{j})^{\mu} \left[ \bar{\phi} \dfrac{\tau_{i}\tau_{j}}{4}\bar{\phi}\right] = \frac{1}{4}\eta^{2} W_{\mu}W^{\mu} \\

W^{3}_{\mu} B^{\mu} \left[ \bar{\phi} \dfrac{\tau_{3}}{4} \bar{\phi} \right] = - \dfrac{1}{4} \eta^{2} W^{3}_{\mu}B^{\mu}
\end{array}
\end{equation}

which becomes:

\begin{equation}
\begin{array}{c}
M^{2}  = \dfrac{1}{2} g^{2} \eta^{2}\\
M_{0}^{2}  = \dfrac{1}{2} (g^{\prime})^{2} \eta^{2}\\
M_{03}^{2}  = -\dfrac{1}{2} gg^{\prime} \eta^{2}
\end{array}
\end{equation}

therefore:

\begin{equation}
\mathcal{M} =  \dfrac{1}{2} \eta^{2}
\begin{pmatrix}
 g^{2} & -gg^{\prime} \\
-gg^{\prime} & (g^{\prime})^{2} \\
\end{pmatrix}
\label{eq::matrix_mass}
\end{equation}

in order to allow the existence of a massless photon $det(\mathcal{M}) = 0$. Knowing that the vacuum configuration in invariant to gauge transformation associated to the electric charge is possible to introduce the massive field $Z_{\mu}$ and the electric field $A_{\mu}$ so that:

\begin{equation}
\begin{array}{c}
Z_{\mu} = \cos\theta W^{3}_{\mu} -\sin\theta B_{\mu}\\
A_{\mu} =\sin\theta W^{3}_{\mu} - \cos\theta B_{\mu}

\end{array}
\label{eq::fields_rotation}
\end{equation}

with $\theta$ knows as the electroweak mixing angle. By using the condition that $A_{\mu}$ in the eigen-vector of \autoref{eq::matrix_mass} with a zero eigen-value:

\begin{equation}
0 = 
\begin{pmatrix}
g^{2} & -gg^{\prime} \\
-gg^{\prime} & (g^{\prime})^{2} \\
\end{pmatrix}
\binom{\sin\theta}{\cos\theta}
=
\binom{g^{2}\sin\theta - gg^{\prime}\cos\theta}{-gg^{\prime}\sin\theta + (g^{\prime})^{2}\cos\theta}
\label{eq::matrix_rotation}
\end{equation}

the equation in solved for:

\begin{equation}
\tan\theta = \dfrac{g^{\prime}}{g}
\label{eq::matrix_solution}
\end{equation}

this condition couples the field $A_{\mu}$ defined in \autoref{eq::fields_rotation} with the electron through the electromagnetic current with

\begin{equation}
g\sin\theta = g^{\prime} \cos\prime = e; 
\end{equation}

The theory symmetry breaking made by Weimber and Salam, based on a symmetric Lagrangian, reproduces the masses spectrum and the couplings on the vectorial fields. All those results has been experimentally confirmed.

\subsection{The fermionic masses}

In order to complete the electro-weak theory the calculation of the fermionic masses is mandatory. Starting with the electron mass taken from the $\mathcal{L}_{\text{m}}$ term of the Glashow theory:

\begin{equation}
\mathcal{L}_{\text{m}} = m_{\text{e}}\bar{e}e = m_{\text{e}}(\bar{e}_{\text{L}}e_{\text{R}} + \bar{e}_{\text{R}}e_{\text{L}})
\end{equation}

Another invariant Lagrangian is obtainable by combining $\mathcal{L}_{\text{m}}$ with the Higgs field, which also has the electro-weak iso-spin of $1/2$. Following the symmetry breaking, $\phi$ gains a constant component, which reproduces the Lagrangian  $\mathcal{L}_{\text{m}}$, while the quantum component of $\phi$ gives raise to a new interaction between $\phi$ and the electron. The invariant Lagrangian becomes:

\begin{equation}
\mathcal{L}_{e\phi} = g_{e} (\bar{l}\phi e_{\text{R}} + \bar{e}_{\text{R}}\phi^{\dagger}l)
\end{equation}

The invariant term $\bar{l}\phi$, under spontaneous symmetry breaking, in the unitary gauge becomes:

\begin{equation}
\bar{l}\phi = \bar{\nu}_{\text{L}}\phi^{+} + \bar{e}_{\text{L}}\phi^{0} = \bar{e}_{\text{L}}\left(\eta + \dfrac{\sigma}{\sqrt{2}}\right)
\end{equation}

obtaining:

\begin{equation}
\mathcal{L}_{e\phi} = g_{e}\eta\bar{e}e + g_{e} \dfrac{\sigma}{\sqrt{2}}\bar{e}e = \mathcal{L}_{\text{m}} + \text{interactions}
\end{equation}

resulting in a electron with mass term:

\begin{equation}
m_{\text{e}} = g_{e}\eta
\end{equation}

\subsection{Cabibbo's Theory}

The concept of quark mixing as consequence of the previously introduced symmetry breaking was introduced by Cabibbo \cite{PhysRevLett.10.531}. In this scheme the left-handed fields are are associated to an electroweak doublet and the right-handed ones to singlets:

\begin{equation}
\binom{u}{d}_{\text{L}}; \quad u_{\text{R}}; \quad d_{\text{R}}; \quad s_{\text{L}}; \quad s_{\text{R}}.
\end{equation}

Cabibbo shows that the symmetry breaking leads to a mixing between $d_{\text{L}}$ and $s_{\text{L}}$ leading to a expression of the charged current in the form:

\begin{equation}
J^{1}_{\mu} + i J^{2}_{\mu}= \bar{u}_{\text{L}}\gamma_{\mu} (\cos\theta d_{\text{L}} +\sin\theta s_{\text{L}})
\label{eq::charg_current}
\end{equation}

including a single electroweak parameter, the angle $\theta$ also known as the Cabibbo angle. By the introduction of the triplet:

\begin{equation}
q_{\text{L}} = 
\begin{pmatrix}
u \\
d \\
s \\
\end{pmatrix}
_{\text{L}}
\end{equation}

\autoref{eq::charg_current} becomes:

\begin{equation}
J^{1}_{\mu} + i J^{2}_{\mu} = \bar{q}_{\text{L}}\mathcal{C}\gamma_{\mu}q_{\text{L}} =  \bar{q}_{\text{L}}
\begin{pmatrix}
0 &\cos\theta &\sin\theta \\
0 &0 &0 \\
0 &0 &0\\
\end{pmatrix}
\gamma_{\mu}q_{\text{L}}
\label{eq::charg_current_matrix}
\end{equation}

However the Cabibbo's theory cannot be included in Glashow-Weimber-Salam electroweak theory because \autoref{eq::charg_current_matrix} would lead to a flavor-changing neutral current, in contrast with what shown by experimental data. This current involves the $\mathcal{C}$ commutator which is not diagonal:

\begin{equation}
\left[\mathcal{C}, \mathcal{C}^{\dagger} \right]
=
\begin{pmatrix}
1 &0 &0 \\
0 &\cos^{2}\theta &-\sin\theta \cos\theta \\
0 &-\sin\theta cos\theta &-\sin^{2}\theta\\
\end{pmatrix}
\end{equation}

The neutral current terms in the form of $\bar{d}_{\text{L}}\gamma_{\mu}s_{\text{L}}$ would produce the following decay $K_{\text{L}} \longrightarrow \mu^{+}\mu^{-}$ at the same rate of $K^{+} \longrightarrow \mu^{+}\nu$

\subsection{The Glashow-Iliopoulos-Maiani mechanism}

The flavor-changing neutral current issue was solved in 1970 by S. L. Glashow, J. Iliopoulos e L. Maiani \cite{Glashow:1970gm} by introducing the existence of a fourth quark, named "charm" quark, with the same electroweak quantum numbers of up quark. With the presence of the charm quark the $s_{\text{L}}$ quark goes into a new doublet as follows: 

\begin{equation}
\binom{u}{d}_{\text{L}}; \quad \binom{c}{s}_{\text{L}}; \quad u_{\text{R}}; \quad d_{\text{R}}; \quad c_{\text{R}}; \quad s_{\text{R}}.
\end{equation}

The $\mathcal{C}$ matrix becomes 4 x 4 and the charged weak current becomes:

\begin{equation}
J^{1}_{\mu} + i J^{2}_{\mu} = \bar{q}_{\text{L}} 
\begin{pmatrix}
0 &U_{\text{GIM}} \\
0 &0 \\
\end{pmatrix}
\gamma_{\mu}q_{\text{L}}
\label{eq::cab_current_gim}
\end{equation}

where:

\begin{equation}
U_{\text{GIM}} =
\begin{pmatrix}
\cos\theta &\sin\theta \\
-\sin\theta &\cos\theta \\
\end{pmatrix}
\end{equation}

Given the $U_{\text{GIM}}$ unitarity, the neutral current now becomes:

\begin{equation}
	\begin{array}{r c l}
		J^{3}_{\mu} = \bar{q}_{\text{L}}\left[\mathcal{C}, \mathcal{C}^{\dagger} \right]\gamma_{\mu}q_{\text{L}} = \\
		\bar{q}_{\text{L}}
		\begin{pmatrix}
			U_{\text{GIM}}U_{\text{GIM}}^{\dagger} &0 \\
			0 &-U_{\text{GIM}}^{\dagger}U_{\text{GIM}} \\
		\end{pmatrix}
		\gamma_{\mu}q_{\text{L}} = \bar{q}_{\text{L}}
		\begin{pmatrix}
			1 &0 \\
			0 &-1 \\
		\end{pmatrix}
		\gamma_{\mu}q_{\text{L}}
	\end{array}
\end{equation}

\subsection{CP violation and Kobayashi-Maskawa theory}

After the introduction if the charm quark and the flavor-changing neutral currents suppression, the missing puzzle piece to a fundamental electroweak theory in 1973 was the natural inclusion of CP violation, observed in the neutral mesons K decays since 1964. 
The starting point is demonstrating that a theory with two doublets the $U_{\text{GIM}}$ matrix can always become real under a phase transformation \cite{Glashow:1970gm}. The easiest way to show it starts from the general definition of the Cabibbo's current:

\begin{equation}
J^{\text{C}}_{\mu} = \bar{u}_{\text{L}}\gamma_{\mu}\left( e^{i\alpha} \cos\theta d_{\text{L}} + e^{i\beta}\sin\theta s_{\text{L}}   \right)
\end{equation}

and observe the two phses can be absorbed by the $d_{\text{L}}$ and $s_{\text{L}}$ definitions. At this point the second row of $U_{\text{GIM}}$ is fixed by the condition of being orthonormal to the first row, obtaining:

\begin{equation}
J^{1}_{\mu} + i J^{2}_{\mu} =  \bar{u}_{\text{L}}\gamma_{\mu} \left(\cos\theta d_{\text{L}} +\sin\theta s_{\text{L}}\right) + \bar{c}_{\text{L}}\gamma_{\mu}e^{i\phi} (-\sin\theta d_{\text{L}} + \cos\theta s_{\text{L}})
\end{equation}

is now possible to absorb the phase $e^{i\phi}$ in the field $c_{\text{L}}$ and obtain the real form of the current as shown in \autoref{eq::cab_current_gim}. Given the number of existing left-handed doublets and right handed singlets N M. Kobayashi and T. Maskawa found out that the minimum case where the quark fields can exist with an irreducible phase is at $N = 3$. Given the existence of ad additional doublet the CP violation can now be included in the Weimberg-Salam theory.

\subsection{Masses and Quark mixing}

Once added all the remaining families the general Lagrangian becomes:

\begin{equation}
\mathcal{L}_{ij} = g^{\text{D}}_{ij}\bar{Q}_{i}D_{j} + g^{\text{U}}_{ij}\epsilon_{ab}\bar{U}_{i}Q^{a}_{j}\phi^{b} + \text{h.c.} \quad (i,j = 1,2,3)
\end{equation}

\begin{equation}
\mathcal{L}_{\text{qH}} = \sum_{ij}\mathcal{L}_{ij}
\label{eq::lagrangian_qH}
\end{equation}

where $Q_{i}$ represent the generic left-handed doublet and $U_{i}$ and $D_{i}$ defines the right handed fields of type \textit{up} (u, c, t) and type \textit{down} (d, s, b). The values of the coupling constants $g^{\text{D}}_{ij}$ $g^{\text{U}}_{ij}$ are complex, violating the CP and T symmetry while keeping intact the CPT one.
The quark masses Lagrangian is obtained by using the Higgs field vacuum value in \autoref{eq::lagrangian_qH}:

\begin{equation}
\begin{array}{r c l}
\mathcal{L}_{\text{qm}} = \bar{D}_{\text{L}}M^{\text{d}}D_{\text{R}} + \bar{U}_{\text{R}}M^{\text{u}}U_{\text{L}} + \text{h.c.} \\
(M^{\text{d}})_{ij} = g^{\text{D}}_{ij}\eta; \quad (M^{\text{u}})_{ij} = g^{\text{U}}_{ij}\eta
\end{array}
\end{equation}

where $M^{\text{u}}$ and  $M^{\text{d}}$ are matrices in the doublets family space and they are usually non-diagonal. This introduces a difference between the electroweak base, defined by the quark fields, and the physical base, defined by the fields that diagonalize the mass matrices and directly associated to the physical particles.
The definition of the physical base pass through a decompositions of the $M^{\text{u}}$ and $M^{\text{d}}$ matrices the following way:

\begin{equation}
M^{\text{u}} = W^{\dagger}m_{\text{u}}Z; \quad M^{\text{d}} = U^{\dagger}m_{\text{d}}V
\end{equation}

with $m_u,d$ as positive and diagonal matrices and W, Z, U, V as unitary matrices. The mass Lagrangian is then written:

\begin{equation}
\mathcal{L}_{\text{qm}} = \bar{D}_{\text{L}}U^{\dagger}m_{\text{d}}VD_{\text{R}} + \bar{U}_{\text{R}}W^{\dagger}m_{\text{u}}ZU_{\text{L}} + \text{h.c.}
\end{equation}

The following transformations:

\begin{equation}
U_{\text{R}} \longrightarrow W U_{\text{R}}; \quad D_{\text{R}} \longrightarrow V D_{\text{R}}
\end{equation}

are a redefinition of the electroweak singlets that can be perform in total respect of the electroweak symmetry. A similar redefinition can be also done for the left-handed doublets:

\begin{equation}
Q = \binom{U_{\text{L}}}{D_{\text{L}}} \longrightarrow ZQ
\end{equation}

so the Lagrangian becomes:

\begin{equation}
\begin{array}{r c l}
\mathcal{L}_{\text{qm}} = \bar{D}_{\text{L}}ZU^{\dagger}m_{\text{d}}D_{\text{R}} + \bar{U}_{\text{R}}m_{\text{u}}U_{\text{L}} + \text{h.c.} = \\
= \bar{D}_{\text{L}}U_{\text{CKM}}m_{\text{d}}D_{\text{R}} + \bar{U}_{\text{R}}m_{\text{u}}U_{\text{L}} + \text{h.c.}
\label{eq::Lqm_Uckm}
\end{array}
\end{equation}

The field shown in \autoref{eq::Lqm_Uckm} have still a weak isospin; however in the new base while the U fields have a diagonal mass matrix the D fields are still misaligned. By defining the transformation:

\begin{equation}
\left(D_{\text{ph}}\right)_{\text{L}} = U^{\dagger}_{\text{CKM}}D_{\text{L}}
\end{equation}

or rather:

\begin{equation}
D_{\text{L}} = 
\begin{pmatrix}
d  \\
s  \\
b \\
\end{pmatrix}
_{\text{L}} =
U_{\text{CKM}}\left(D_{\text{ph}}\right)_{\text{L}} = U_{\text{CKM}}
\begin{pmatrix}
d_{\text{ph}}  \\
s_{\text{ph}}  \\
b_{\text{ph}} \\
\end{pmatrix}
_{\text{L}} ;
\end{equation}

\begin{equation}
D_{\text{ph}} = \left(D_{\text{ph}}\right)_{\text{L}} + D_{\text{R}}
\end{equation}

The mass Lagrangian is now finally diagonal:

\begin{equation}
\begin{array}{r c l}
\mathcal{L}_{\text{qm}} = \bar{D}_{\text{L}}ZU^{\dagger}m_{\text{d}}D_{\text{R}} + \bar{U}_{\text{R}}m_{\text{u}}U_{\text{L}} + \text{h.c.} = \\
= \bar{D}_{\text{L}}U_{\text{CKM}}m_{\text{d}}D_{\text{R}} + \bar{U}_{\text{R}}m_{\text{u}}U_{\text{L}} + \text{h.c.} = \\
=  \left(\bar{D}_{\text{ph}}\right)_{\text{L}}m_{\text{d}}D_{\text{R}} + \left(\bar{U}_{\text{ph}}\right)_{\text{R}}m_{\text{u}}\left(U_{\text{ph}}\right)_{\text{L}} + \text{h.c.} = \\
= \bar{D}_{\text{ph}}m_{\text{d}}D_{\text{ph}} + \bar{U}_{\text{ph}}m_{\text{u}}U_{\text{ph}}
\label{eq::Lqm_Uckm_diagonal}
\end{array}
\end{equation}

the matrix $U_{\text{CKM}}$ is indeed the Cabibbo-Kobayashi-Maskawa matrix and it's misalignment breaks the electroweak symmetry.

\subsection{Standard Model limitations}

Over the decades the Standard Model has been successful in predicting particle existence and properties at the electroweak scale. However a few questions remain:

\begin{itemize}
	\item The Standard Model doesn't include gravity, therefore it can't be a complete description of nature;
	\item Although the successful representation of the strong force by the \text{SU}(3), the Standard Model doesn't have the unification of the force coupling constants at the Planck scale.
	\item Astrophysical researches give evidence to the existence of a much greater amount of matter in the universe that can be explained by baryonic matter \cite{deBoer:2005tm}, known also under the name of \textit{dark matter}. The only Standard Model candidate that has properties similar to Dark Matter are neutrinos, however le low rate of neutrino production can't justify the measured dark matter amount;
	\item Neutrinos are considered massless by the Standard Model. Recent experiments showed that neutrinos have masses \cite{Fukuda:1998mi};
	\item In the Standard Model, the quantum corrections to the Higgs mass are quadratically divergent. If the Standard Model is assumed to be valid up to the Planck scale, these corrections are huge compared to the physical Higgs mass.
\end{itemize}

\clearpage

\section{Supersymmetry}

Supersymmetry is one of the most intriguing and fundamental concepts in modern theoretical particle physics. It arises naturally from the combination of the two cornerstones of 20th century physics: quantum mechanics and relativity. Supersymmetry is the unique symmetry that relates the two fundamental kinds of particles: bosons, which act as the carriers of forces, and fermions, which act as the constituents of matter. Supersymmetry transformations are in a sense like the square roots of the coordinate system transformations in special relativity, and consequently supersymmetric quantum field theories have very special, improved properties, compared to ordinary relativistic quantum field theories. If supersymmetry is realized in nature, every fermion in the SM must have a bosonic partner particle and vice versa. No such superpartner particle has been observed so far but there are more and more indications that these particles might show up at the LHC experiments.

\begin{itemize}
	\item introduction of supersymmetric transformation
	\item Supersymmetric generators properties
	\item supersymmetric lagrangian
	
\end{itemize}

\subsection{Motivations}

\begin{itemize}
	\item Dark matter candidate
	\item Higgs corrections \autoref{fig:higgs_loop}
	\item Gravity inclusion
\end{itemize}

\begin{figure}[tbh!]
	\centering
	\begin{tabular}{cc}
		\includegraphics[width=0.75\textwidth]{theory/pics/higgs_loop.png}
	\end{tabular}
	\caption{In SUSY, the correction to Higgs mass by the top quark (L) is inherently cancelled by the contribution from the top quark's supersymmetric partner, the stop (R).}
	\label{fig:higgs_loop}
\end{figure}

\begin{figure}[tbh!]
	\centering
	\begin{tabular}{cc}
		\includegraphics[width=0.75\textwidth]{theory/pics/SUSY_naturalness.png}
	\end{tabular}
	\caption{Cartoon illustration of the mass scales for various sparticles dictated solely by electroweak naturalness with sensitivity parameter $\Delta \lesssim 10$.}
	\label{fig:SUSY_naturalness}
\end{figure}


\begin{figure}[tbh!]
	\centering
	\begin{tabular}{cc}
		\includegraphics[width=0.75\textwidth]{theory/pics/SUSY_particles_table.png}
	\end{tabular}
	\caption{SUSY particles in MSSM~\protect\cite{Martin:1997ns}}
	\label{fig:SUSY_particles_table}
\end{figure}

\subsection{The MSSM}

\begin{itemize}
	\item R-parity
	\item further assumptions
\end{itemize}

\subsection{Particle content}

\begin{itemize}
	\item ewkino mixing resultin in charginos and neutralinos
	\item multihiggs, explain why there are multiple ones
	\item chiral supermultiples
	\item gauge supermultiplets
	
\end{itemize}

\subsection{SUSY generic signatures at the LHC with charginos and neutralinos}
\begin{itemize}
	\item cascade decays
	\item 3-body decays and mass comfiguration
\end{itemize}
