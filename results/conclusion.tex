This thesis presented a search for non-colored supersymmetric particles in the vector-boson fusion (VBF) topology using data corresponding to an integrated luminosity of $19.7\fbinv$ collected with the CMS detector in proton-proton collisions at $\sqrt{s} = 8\tev$. This is the first search for SUSY in the VBF topology and makes use of events in eight different final states covering both same and opposite sign dilepton pairs. The leptons considered are electrons, muons, and hadronically decaying $\tau$ leptons. The VBF topology requires two well-separated jets that appear in opposite hemispheres, with large invariant mass \mjj. 

The work covered in this thesis predominantly features the background estimation technique for one of the most challenging channels of this search, the same sign di-\hadtau channel. The QCD background is considered the most relevant contribution and is estimated via a so-called ABCD method utilizing the inversion of VBF jet selection and \hadtau isolation requirements. This technique was used to predict the QCD background contribution in the signal region to be:

\begin{equation}
N^{QCD}_{SR} = 7.59\pm0.92(stat.)^{-0.42(MC)+1.34(\tau iso)+0.20(MET)}_{+0.35(MC)-0.58(\tau iso)-0.20(MET)}
\end{equation} 

This number is in full agreement with the measured number of data events. The combined observed \mjj distributions do not reveal any evidence for new physics. The results are used to exclude a range of \charginopm and \neutralinotwo gaugino masses. For models in which the \neutralinoone lightest supersymmetric particle mass is zero, and in which the \charginopm and \neutralinotwo branching fractions to $\tau$ leptons are large, \charginopm and \neutralinotwo masses up to $300\gev$ are excluded at the $95\%$ confidence level. For a compressed-mass-spectrum scenario, in which $m(\charginopm) - m(\neutralinoone) = 50\gev$, the corresponding limit is 170\gev. While many previous studies at the LHC have focused on strongly coupled supersymmetric particles, including searches for charginos and neutralinos produced in gluino or squark decay chains, and a number of studies have presented limits on the Drell-Yan production of charginos and neutralinos, this analysis obtains stringent limits on the production of charginos and neutralinos decaying to $\tau$ leptons in compressed mass spectrum scenarios defined by the mass separation $\Delta m = m(\charginopm) - m(\neutralinoone) < 50\gev$.

The sensitivity of the di-\hadtau channel is predominantly limited by the di-\hadtau trigger efficiency. Even though the hadronic channel has the best branching ratio fraction from the $\tau$ lepton decay, the overall contribution to the combined exclusion limit is small compared to the one of the other lepton channels. With the aim to search for a better trigger strategy for the next run of the analysis, a sensitivity study is performed for the same topology of events using 13\tev simulated data and an integrated luminosity of 85\invfb. The first part of this sensitivity study made use of a looser event selection compared to the one used with 8\tev data, making no requirements on the momentum of the reconstructed tau leptons  $\pt\left(\hadtau\right)$, the missing transverse momentum\met or the invariant mass of the dijet candidate\mjj. This study demonstrates the potential for excluding models with \charginopm masses below $m\left(\charginopm\right) = 380\gev$, particularly for compressed spectra SUSY scenarios. The last part of the sensitivity study has the aim of applying an event selection that falls in line with the available trigger choices the CMS detector for the second part of the data taking period. The most promising and realistic trigger choice is the one that applies an online selection over the VBF properties of the event. By incorporating this trigger into the analysis, the potential to exclude models with \charginopm masses below $m\left(\charginopm\right) = 280\gev$ for compressed spectra SUSY scenarios is greatly increased.